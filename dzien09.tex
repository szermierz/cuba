
\chapter[Brakuje tylko autostopowiczów z bananami]{Dzień 9 - Brakuje tylko autostopowiczów z bananami}

Wieczorem, gdy już spałem, lokalny bar obudził mnie muzyką.
Udało mi się przespać parę godzin, ale później nastała senna bessa.
Wspomniane przeze mnie wcześniej okna nic a nic nie chroniły nas przed dźwiękiem, przez co rano wstałem niezbyt wyspany.
Miałem tylko nadzieję, że Kozi spał lepiej, bo czekała nas 8smio godzinna jazda, a to on jest kierowcą.
Oczywiście wynajmując auto zdecydowaliśmy się na dopłatę, abym i ja mógł je prowadzić, za to póki co nie skorzystaliśmy z tej możliwości.
Koziemu dobrze wychodziły nagłe hamowania przed dziurami, a zwycięskiego składu się nie zmienia.
\par Śniadanie u Mariji nie było specjalnie bogate, za to całkiem smakowite.
Każdy z nas dostał po kruchym ciastku, które miało przyjemny smak pochodzący z brązowego cukru.
Na talerzu owoców natomiast znajdowały się plasterki w kształcie gwiazdek.
Spytałem się Mariji jak nazywa się ten owoc i jego hiszpańska nazwa to alberoa.
Jeśli dobrze pamiętam, po angielsku nazywa się starfruit, za to po polsku jeszcze inaczej: karambola. Sam owoc ma bardzo subtelny smak albo po prostu nam trafił się taki mało intensywny egzemplarz.
\par Przez większość nocy padał intensywny deszcz.
Rano już się rozpogodziło, ale pierwszy odcinek dzisiejszej długiej trasy wiedzie przez góry i szczerze mówiąc bałem się, że może się na nas osunąć jakaś - mniejsza lub nie - lawina.
Poruszyłem ten temat z Kozim i wspólnie doszliśmy do wniosku, że istnieje takie ryzyko, ale do osunięć doszło już raczej w nocy, za to jadąc drugą trasą: bezpośrednio na zachód, narażamy się na utonięcie w błocie, co mogłoby poważnie zagrozić naszej podróży.
Zapłaciliśmy i pożegnaliśmy się.
Wsiedliśmy w samochód i pojechaliśmy w kierunku stacji benzynowej, gdzie z powodzeniem napełniliśmy bak.
Odcinek górski faktycznie był w gorszym stanie niż dzień wcześniej.
Kilkukrotnie napotkaliśmy zablokowany jeden z pasów, za to udało nam się pokonać go całego.
W trakcie jazdy mijaliśmy przydrożnych handlarzy wymachujących do nas np.
kiścią bananów.
Widok ten towarzyszył nam przez całą Kubę, ale tym razem przeszło mi przez głowę pytanie: A co jeśli oni nie chcą nam tego sprzedać, tylko oferują nam to jako opłatę za podwózkę?
Byłem w stanie zrozumieć próbę handlu w wielu miejscach, ale tutaj w górach ciężko mi uwierzyć, że ktoś zatrzymałby się je kupić.
Nie dlatego że sam taki zakup jest dziwny i właściwie niebezpieczny, ale dlatego że ani razu nie widziałem, żeby ktoś się zatrzymał dokonać wymiany, więc statystycznie rzecz ujmując rzadko jeżdżące tą trasą samochody zatrzymują się w przybliżeniu...
nigdy.
Przeszło mi też przez myśl zatrzymać się i zapytać, ale nie chciałbym zrobić takiemu człowiekowi nadziei na jazdę klimatyzowanym pojazdem z nami, a potem potraktować go jak eksponat z zoo i poprosić tylko o wytłumaczenie swojego celu.
Krótko po pokonaniu odcinka górskiego przyszło nam po raz pierwszy na Kubie zatrzymać się przed przejazdem kolejowym.
Nie był on oczywiście zamknięty szlabanami, za to pociąg już z daleka nieustannie trąbił, więc nie sposób było go nie zauważyć.
Pociąg realizował transport różnych towarów, większość wagonów była drewnianą skrzynią z zaokrąglonym dachem, więc ciężko mi zgadywać co mogło być w środku.
Były też dwa wagony cysterny, kilka pustych, jeden z żywcem (zwierzętami hodowlanymi, nie piwem).
Kuba była jednym z pierwszych krajów obu Ameryk, gdzie uruchomiona została sieć kolejowa.
Na całej Kubie znajdują się pomniki przedstawiające zabytkowe lokomotywy, które jako pierwsze przemierzały szyny tego kraju.
Ciekawe, kiedy znów będziemy widzieli jakiś pociąg jadący przez Kubę.
W związku z tym, że dzisiejsza trasa była zdecydowanie dłuższa niż poprzednie, zdecydowaliśmy się zaburzyć rytm dnia i obiad zjeść na postoju.
W przeciwnym wypadku udałoby się nam to dopiero w okolicy 18stej - po dotarciu na sam koniec trasy.
Zatrzymaliśmy się na samym początku obwodnicy Las Tunas, w knajpie o nazwie El Ranchon.
Była to dosyć spora buda ze słomianym dachem wznoszącym się na drewnianych palach ponacinanych jak kiełbaski z grilla (co jak mniemam było elementem ozdobnym).
Menu wywieszone było na słupie i zawierało dwa dania i trzy przystawki (w tej kolejności, od góry do dołu).
Wszystkie pozycje były wypisane jedynie po hiszpańsku, więc wyciągnąłem słownik i podjąłem się tłumaczenia dań.
Pierwsze okazało się potrawką z krewetek.
Drugie natomiast czymś ze świni, więc bez wahania wybrałem drugą pozycję.
Kozi natomiast poprosił o danie spod lady: stek wołowy.
Dla pewności przetłumaczyłem też resztę nazwy mojej potrawy i jak się okazało była to...
świńska noga.
Z jednej strony mogło się to okazać czymś w rodzaju naszej polskiej świńskiej nogi w kapuście, tylko pewnie bez kapusty.
Z drugiej strony niezbadane są ścieżki kubańskich wymysłów kulinarnych, więc właściwie mogło to być cokolwiek.
Nie wiem która z tych opcji przerażała mnie bardziej.
Na szczęście okazało się, że moje danie było zwykłymi plastrami schabu w cebuli, podanymi z fasolką, awokado, chipsami bananowymi i brązowym ryżem.
Posiłek był pyszny!
W trakcie jedzenia borykaliśmy się z niezwykle silnym wiatrem, który na terenie całej knajpy targał ze sobą serwetki, puste puszki, a chwilowo i puste butelki.
Silny wiatr i głośna muzyka właściwie wykluczyła jakąkolwiek rozmowę, więc mogliśmy odezwać się do siebie dopiero po skończonym posiłku.
Gdy to nastąpiło, zapłaciliśmy i zebraliśmy się do auta.
\par Ruszyliśmy w dalszą trasę, kierując się na obwodnicę Las Tunas.
Ku naszemu zdziwieniu już kawałek dalej zabrakło asfaltu.
Przez najbliższe 5km przyszło nam jechać po żwirze.
Droga chyba była po prostu w remoncie, co nie było zaznaczone na naszych mapach.
Wcześniej jadąc przez Las Tunas wypadło nam przemierzyć je przez samo centrum, które wyglądało ślicznie i szczerze mówiąc byłem trochę rozczarowany tym, jak tym razem poprowadziła nas nawigacja.
Na szczęście zaraz za miastem asfalt powrócił i mogliśmy normalnie (jak na warunku Kubańskie, czyli omijając wszędobylskie dziury drogowe) jechać.
Celem naszej podróży było La Boca, mała miejscowość wybrana trochę przez położenie palca na mapie - potrzebowaliśmy umieścić w okolicy tego miejsca nocleg, aby dziennie pokonywana trasa nie była zbyt długa. La Boca wygrało w plebiscycie na najlepszą lokalizację zarówno pod kątem odległości do poprzedniego i następnego przystanku, jak i bliskości do Oceanu Atlantyckiego.
\par Z każdym pokonanym przez nas kilometrem jakość dróg się pogarszała.
Apogeum nastąpiło, gdy z trasy na Hawanę odbiliśmy na północ.
Po oznaczeniach na mapie podejrzewam, że drogi na Kubie można podzielić na 5 klas.
Klasa drogi jest mniej więcej powiązana ze stanem jej nawierzchni, ale zdarzają się brutalne wyjątki.
Najlepsze jakościowo są autostrady, następnie drogi oznaczone jako CC, kolejno trzecia klasa CN, czwarta: drogi regionalne i piąta - drogi lokalne.
Pierwsze trzy kategorie oznaczone były na naszej mapie kolorem pomarańczowym, kategoria czwarta białym, piąta szarym.
Odcinek, którym przyszło nam tu jechać był drogą czwartej kategorii.
O ile na początku był jeszcze asfalt z dziurami tak głębokimi, że jeden nieostrożny ruch mógłby uszkodzić nam zawieszenie, tak po chwili jechaliśmy już po czymś, co chyba kiedyś było asfaltem, a dziś było pozostałością po jego sfrezowaniu.
Liczba dziur się spotęgowała.
Cały ten teren przypominał poligon doświadczalny na którym testowane były manewry bombardowania.
Dziury o średnicach z zakresu 0,5m do 2m, głębokie na 15-30cm.
Niemalże jedna przy drugiej.
Czasami obok tej „drogi” pojawiał się dodatkowy pas ziemi, którym mogliśmy przejechać około 50m bez obawy o utratę koła.
Blisko 20km pokonaliśmy iście żółwim tempem, nieustannie snując żarty o Kubańskich drogach.

\szQuote{Brakuje tylko autostopowiczów z bananami.}
\noindent Na szczęście, gdy tylko dojechaliśmy do północnej nitki komunikacyjnej, wjechaliśmy na jednolity asfalt.
Jednolity - bo dziury w nim załatane były asfaltem i taka droga towarzyszyła nam już do samego La Boca.
Na miejsce dotarliśmy w okolicy godziny 19stej.
\par Gdy wysiedliśmy odniosłem przez chwile wrażenie, że zrobiło się jakoś zimno, ale po chwili dotarło do mnie, że chyba po prostu ochłodziło się do 25 stopni.
Ruszyliśmy pieszo w poszukiwaniu noclegu, początkowo bez żadnych sukcesów.
Pierwsze dwa znalezione przez nas hostele okazały się ślicznymi domkami, niestety opustoszałymi.
Idąc dalej nasz wzrok przykuło ogrodzenie ze stali nierdzewnej, a gdy podeszliśmy okazało się, że przy furtce jest nawet działający dzwonek do drzwi.
Zadzwoniliśmy i po chwili wyszedł nas przywitać mężczyzna, który zaoferował nam jeden z dwóch pokoi na piętrze.
Pokój nie był jakoś specjalnie przytulny czy ładnie ozdobiony, za to bardzo praktyczny.
W każdym z nich okna okryte były moskitierami, a my mogliśmy nawet zaparkować samochód na zamykanym podwórzu, co było niespotykanym dotąd na Kubie komfortem.
W związku z tym, że mężczyzna był łysy, postanowiliśmy nadać mu imię Johnny, na cześć pewnej gwiazdy filmowej.
\par Rozpakowaliśmy się i póki jeszcze było ciepło ruszyliśmy nad ocean.
Plaża okazała się w dużej mierze pokryta zbrązowiałymi liśćmi, więc przespacerowaliśmy cierpliwie całkiem spory jej kawałek, aby znaleźć fragment z możliwie małym nagromadzeniem tych paprochów, bo ich ilość na brzegu była blisko związana z ilością poruszającą się w wodzie z każdą falą.
Nigdzie nie było idealnie, ale udało nam się znaleźć możliwie czysty odcinek.
Właśnie tam rozkoszowaliśmy się widokiem zachodzącego słońca.
Woda była cieplejsza niż w Morzu Karaibskim, ale też nie taka czysta (i nie chodzi tu o liście, a o piasek i muł unoszący się z każdym pływem).
Możliwe, że była to kwestia piaszczystej plaży zamiast kamienistego brzegu.
Pływaliśmy w Oceanie Atlantyckim dosyć długo, jak tylko było to możliwe, aż zrobiło się całkiem ciemno.
Za to, gdy wracaliśmy, uczepiły się nas komary - całą chmarą, która podążała za nami aż do pokoju.
O dziwo tym razem dotkliwie pogryzły Koziego, za to mnie zostawiły w spokoju.
Dowiedziałem się także, że komary z Cienfuegos pogryzły wtedy tylko mnie, za to Kozi nie miał ani ugryzienia.
Widocznie bardzo ostrożnie i zależnie od regionu dobierają sobie swoje ofiary.