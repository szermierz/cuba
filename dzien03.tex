
\chapter[Słowo Cuba oznacza miejsce, w którym są góry i żyzne ziemie]{Dzień 3 - Słowo Cuba oznacza miejsce, w którym są góry i żyzne ziemie}

Postanowiliśmy, że każdemu właścicielowi pokoju, który wynajmujemy, nadawać będziemy imiona.
Pierwszy z pokoi, ten w Hawanie należał do Gonzalesa.
Drugi z kolei, ten tutaj - w Viñales, należy do faceta, którego ochrzciliśmy imieniem Pacho, gdzie „ch” wymawiane jest jako zgłoska pomiędzy „cz” i „ć”.
Pacho był wobec nas bardzo uprzejmy i wczoraj zaoferował nam darmowy drink, ale odmówiliśmy.
Dziś rano zostało nam przygotowane śniadanie z talerzem bananów, talerzem czerwonego melona i ananasów oraz trzecim talerzem z arbuzami i różowym melonem.
Wszystkie owoce były pokrojone tak, aby stanowić jeden kęs.
Co więcej, dzisiejszy poranek był tym momentem, w którym zdałem sobie sprawę z obecności dwóch rodzajów czerwonego melona (oczywiście żaden z nich nie jest melonem, ale roboczo przyjmują tę nazwę, z powodu mojej nieznajomości ich faktycznej tożsamości gatunkowej).
Jednym rodzajem melona jest czerwony.
Nie ma pestek i smakiem oraz konsystencją przypomina gruszkę (oczywiście bez samego smaku gruszki.
Chodzi o słodkość i kremowość gruszki.
Zapach gruszki wywoływany jest przez ester jednego z najprostszych kwasów karboksylowych, jedno lub dwuwęglowego, nie pamiętam dokładnie którego.
Tutaj nie ma go ani krztyny).
Drugi z kolei jest różowy i posiada dużo twardych pestek.
To właśnie z tego różowego dwukrotnie piliśmy smoothie w Hawanie.
Do śniadania dostaliśmy też ciepłe mleko, jajecznicę, sucharki, masło, żółty ser i jakąś mielonkę pociętą w trójkąty, której żaden z nas nie spróbował.
Banany były nieobrane - widać było, że są dojrzałe, ich skórka była pociemniała.
Po śniadaniu wsiedliśmy w samochód i pojechaliśmy na plantacje tytoniu, gdzie umówieni byliśmy na przejażdżkę konną.
Na miejscu przywitał nas facet, który wyglądał jak kowboj z łesternu.
Zamówiona wycieczka miała być dwuosobowa (plus przewodnik), bo w ten sposób moglibyśmy dowolnie zwalniać jazdę, by porobić zdjęcia.
Z tego powodu odbywała się godzinę przed pierwsza oferowana wycieczką: o 9 rano.
Tak przynajmniej się umówiliśmy.
Widocznie nie tylko my chcieliśmy z tego skorzystać, bo wraz z nami na zbiórce stawiła się para Belgów.
Pojechaliśmy na kubańskie rancho, gdzie każdy z nas otrzymał konia do jazdy.
Ten mój nazywał się Pahirito, natomiast Koziego Caramelo, przynajmniej fonetycznie.
Ciężko mi zgadywać jaka jest pisownia tych imion.
Nasz przewodnik zarzekał się, że konie są „semi-automatic”, ale nie jestem pewien czy zrozumiał mój żart, że na pewno sobie poradzę, w końcu na co dzień jeżdżę na automatycznej skrzyni biegów.
Wycieczka odbywała się po Kubańskim Parku Narodowym.

\szQuote{Słowo Cuba oznacza miejsce, w którym są góry i żyzne ziemie.}
\noindent Park ten pełen jest spektakularnych widoków dzikiej natury, gór, palm, ścieżek z czerwonej ziemi, zieleni dookoła nich.
Na Kubie uprawiane są trzy rodzaje ziemniaka: pataty, manioki oraz trzeci rodzaj, którego nazwy nie pamiętam, ale jego krzaki mają liście łudząco podobne do charakterystycznego symetrycznego siedmiolistnego zioła z ząbkowanymi krawędziami.
Mijaliśmy plantacje każdego z nich (ziemniaków!), jak i pola trzciny cukrowej, która wygląda identycznie do przerośniętej kukurydzy, tylko bez kukurydzy.
Nasz przewodnik nieustannie popędzał konie krzycząc ich imiona w losowej kolejności.
Podejrzewam, że ma to sprawić turystom radość, podobnie jak nieustannie trąbiący taksówkarze.
Pierwszym przystankiem wycieczki była plantacja kawy.
Zatrzymaliśmy się, zsiedliśmy z koni i rozprostowaliśmy nogi.
\par Krzewy kawowe owocują między wrześniem a grudniem.
W tym czasie regularnie zbierane są tu owoce kawowca, które stają się czerwone - dojrzałe.
Wśród krzewów sadzi się drzewa nazywane przez Kubańczyków „golasami”, ponieważ posiadają liście tylko w bardzo krótkim okresie, gdy kawa woli mieć mniej światła i gubią je, kiedy kawowce potrzebują go więcej.
Na terenie parku narodowego pracownicy nie mogą używać żadnych nowoczesnych maszyn, w tym ciągników rolniczych, więc kawa zbierana jest ręcznie.
Dzięki temu park widnieje na liście światowego dziedzictwa kulturowego Unesco.
Owoce kawowca miażdżone są w ogromnym drewnianym moździerzu o nazwie „pilon”, którego pomysł pochodzi z Afryki.
Miażdżenie nie polega natomiast na tym samym, co przypraw w klasycznym moździerzu.
Tu wystarczy jedno uderzenie drewnianego młota, aby owoce pękły, dzięki czemu można łatwo ręcznie wydzielić ziarno.
Kubańczycy mają swój własny taniec, który towarzyszy im przy ugniataniu owoców, nazywają go Pilon Dance.
Na odwiedzonej przez nas plantacji uprawiany jest gatunek arabica.
Na Kubie uprawia się również drugi gatunek kawy - robusta.
Arabica jest aromatyczna, za to robusta jest bardziej kwaśna.
Ziarna kawy następnie są palone w stalowym pudle.
W zależności od rozmiaru pudełka oraz czasu palenia uzyskuje się różne efekty końcowego bukietu aromatycznego kawy.
Uprażone ziarna łatwo kruszą się między zębami i szczerze mówiąc wydają się świetną przekąska do chrupania.
Być może istnieją powody, dla których nie powinno się ich jeść w ten sposób (bo jakby nie patrzeć nie robi się tak na co dzień) - natomiast nasza kilkuziarnowa degustacja nie pozwoliła nam ich odkryć, o ile istnieją.
Za to dowiedziałem się nareszcie, jak nazywają się kubańskie melony!
Czerwony z nich to papaja, natomiast różowy to guawa.
Równolegle z produkcja kawy zbierany jest miód z kwiatów kawowca.
Kwiaty te są koloru białego, za to produkowany z nich miód jest mniej słodki, delikatniejszy.
Do tropienia uli wykorzystywane są psy.
W ulu wycinany jest okrągły otwór i część zgromadzonego miodu jest wybierana.
Tak oszczędne zbieranie pozwala pszczołom odbudować ul, przez co nie stają się narażone na wyginiecie z powodu utraty domu.
Trzecim i ostatnim produktem plantacji kawy jest czterdziesto procentowy rum.
Podobnie jak każdy inny rum na Kubie, tak i ten produkowany jest z cukru trzcinowego, lecz tutaj dodatkiem są kwiaty kawowca.
W ten sposób jest to hybrydowe rozwiązanie cukrowo-owocowe, jedyne takie w całym kraju.
Zostaliśmy poczęstowani każdym z lokalnych produktów, lecz zamiast pitnej kawy dostaliśmy jedynie jej ziarna - stąd pomysł na pochrupanie ich.
Odprowadzająca nas Pani zażartowała sobie, że na Kubie Polacy i Rosjanie znani są z ilości alkoholu jaki piją.
Z przyjemnością odbiliśmy ten żart, że u nas również jesteśmy z tego znani.
Kieliszek (wykonany z bambusa!) rumu wypiliśmy na jeden raz, bez popicia.
Na ten widok dwójka Belgów została poinstruowana, aby wypić go powoli.
Nie wiem, czy miało to oznaczać, że ominęła nas część walorów smakowych, czy była to raczej obawa, że Belgowie się zakrztuszą, i chyba wole nie szukać odpowiedzi na to pytanie.
Zostaliśmy również poczęstowani sokiem z trzciny cukrowej, który smakował niemalże identycznie jak mleko z cukrem.
W napój włożona była laska trzciny cukrowej, którą można było pogryźć (i nie zalecane jest połykać, ponieważ jest bardzo włóknista i podejrzewam, że nasz ludzki układ trawienny mógłby sobie z tym nie poradzić).
Sok z trzciny cukrowej smakował dużo bardziej jak trzcina cukrowa niż sama trzcina cukrowa.
\par Po chwili na dokończenie napoju wróciliśmy na konie i ruszyliśmy w dalszą trasę.
Jeśli na chwile zdążyliśmy zapomnieć o tym, jak piękne są widoki Narodowego Parku Kubańskiego, to teraz znów uderzyło w nas to z pełną siłą.
Nasza trasa poprowadzona była przez płytką rzekę, a gdy się do niej zbliżaliśmy, przewodnik zażartował żebyśmy uważali na anakondy i krokodyle.
Koń krocząc przez wodę chlapie wszędzie naokoło.
Nie jest to problemem, ponieważ na pasażera nie pada przez to ani kropla wody czy błota.
Jest tu jednak pewne uogólnienie: na pasażera konia, który akurat chlapie.
Pasażerowie koni dookoła są ochlapywani.
Jazda na koniu przypomina mi jazdę samochodem na autopilocie.
Koń sam zwalnia przed każdą dziurą, nie urwie koła na nierównościach.
To bardzo wygodne rozwiązanie, jadąc na zwierzęciu można mu zaufać.
Dla mnie najbardziej przerażające było zjeżdżanie z góry, po nierównej ziemi.
Sam ostrożnie i powoli schodziłbym taką trasą, za to koń poruszał się swoim zwykłym marszem, w dodatku mając na plecach pasażera.
Następny postój zrobiliśmy nad jakimś jeziorem - w tym momencie pisząc to żałuję, że nie sprawdziłem na mapie jak się nazywało.
Przewodnik pokazał nam powoli sposób w jaki przywiązał uprząż konia do...
stojaka na konie?
Takiego płotku, przy którym stały i do którego były przywiązane.
Węzeł ten miał charakter teleskopowy, to znaczy można było dołożyć dowolną liczbę powtarzających się elementów węzła, co skracało pozostałą, wiszącą w powietrzu linę.
W pobliżu jeziora był bar, w którym zamówiliśmy sobie zapiekanki z jajkiem.
Jedliśmy je ostrożnie, nie dotykając rękami, ponieważ każdy z nas wcześniej dotykał koni.
Gdy zjedliśmy i wypiliśmy zamówione lemoniady z kruszonym lodem, ruszyliśmy dalej.
Nasz przewodnik kazał nam na siebie czekać tym, że kontynuował jakaś z pewnością fascynująca opowieść, opowiadaną swoim znajomym, z której ja nie rozumiałem ani słowa.
Gdy już skończył, pożegnał się z nimi i ruszyliśmy w ostatni tour, z powrotem na rancho.
Oddaliśmy konie i wsiedliśmy do auta, po czym wróciliśmy do pokoju.
\par Po orzeźwiającym prysznicu, zapakowaliśmy się znów w auto i pojechaliśmy w miejsce, które Hose pokazał nam na mapie jako wypożyczalnia łódek.
Na niebie robiło się coraz ciemniej, pojawiały się chmury zwiastujące burze.
Parking przy „wypożyczalni łódek” okazał się być płatny, za to dzięki temu od poborcy opłat dowiedzieliśmy się, gdzie ta atrakcja właściwie się znajduje, bo sam parking znajdował się przy zwykłym przydrożnym barze.
Zgodnie ze wskazówkami przeszliśmy 200 metrów dalej, gdzie znajdowała się kolejna kasa biletowa, z opisem określającym ją jako „wycieczkę po jaskini”.
Kupiliśmy dwa bilety i podążając za drogowskazami odnaleźliśmy wejście do jaskiń.
W środku roiło się od zdumiewająco pięknych skał osadowych.
Chodnik oznaczony był przez dwie żółte linie, których położenie wybrano tak, by zagwarantować bezpieczeństwo odwiedzających.
W moich oczach wykracza to daleko poza kubańskie standardy bezpieczeństwa.
Możliwe, że kiedyś wydarzył się tu jakiś wypadek.
Albo kilka.
Nasza piesza wędrówka przez jaskinie kończyła się schodami w dół - w kierunku rzeki.
Tam cyklicznie podpływała łódka napędzana motorem, zbierająca posiadających bilet turystów.
Wsiedliśmy i popłynęliśmy w głąb jaskini.
Motorniczy pokazywał nam zielonym laserem istotne punkty na ścianach, ale o ile na początku spodziewałem się, że są to faktyczne elementy historyczne jaskini, tak sądząc po „wiszącej świni” i „ogromnej twarzy”, były to chyba jedynie skamieliny o różnych kształtach, a nie faktyczne zachowane w kamieniu obiekty.
Wśród nich był także konik morski, palma i kilka innych których nie wychwyciłem z angielskiego opisu ubarwionego hiszpańskim akcentem.
Najbardziej zjawiskowym momentem wycieczki było wypłynięcie z jaskini - przed nami przez szczelinę wyłaniała się jasna furtka na zielony botaniczny świat zewnętrzny.
Zaraz za nią wycieczka się kończyła.
Jaskinie były przepiękne, lecz spodziewaliśmy się, że całość potrwa dłużej niż 30 minut.
Właściwie to spodziewałem się, że wypożyczymy łódź, którą samowolnie popłyniemy po jakimś jeziorze.
\par Po wyjściu z przystani zaczęło grzmieć, a potem stopniowo kropić.
Dotarliśmy do samochodu i pojechaliśmy zatankować.
Jutro rano planujemy wyjechać do Cienfuegos.
Na stacji benzynowej okazało się jednak, że nie wolno nam tam tankować.
Obsługa stacji wskazała nam stacje, na której będzie to możliwe: 25km dalej w Pinar del Rio.
Nasza trasa i tak przebiega przez ten punkt, ponieważ tam zaczyna się autostrada A4, którą planujemy jechać.
Paliwa mamy oczywiście dość, natomiast stacje benzynowe na Kubie nie są specjalnie częste, do tego nie wiemy które z nich (widocznych dla nas na mapie) w ogóle mogą nam sprzedać paliwo.
Jest to jakiś punkt ryzyka całej wycieczki.
W końcu ciężko będzie nam poruszać się po Kubie, gdy stacje konsekwentnie będą nam odmawiać sprzedaży.
Wróciliśmy zatem do pokoju, dokonaliśmy szybkiego przepakowania i ruszyliśmy pieszo na kolacje.
Tym razem nie chcąc trafić na słaba restauracje tak jak wczoraj - poszliśmy prosto w miejsce, gdzie ostatnio piłem Tę Wyborną Pinacoladę.
W drodze tam rozpadał się deszcz, ale nie była to żadna ulewa, a temperatura powietrza nawet w czasie opadu była bardzo przyjemna.
Śmialiśmy się na głos widząc uciekających Kubańczyków, że chcielibyśmy mieć w Polsce takie deszcze.
 Mokrzy usiedliśmy w restauracji i zamówiliśmy po drinku, w międzyczasie wertując menu.
Ja na kolacje zamówiłem smażonego kurczaka, Kozi natomiast stek wołowy.
Wizualnie jego dzisiejszy posiłek można byli porównać z wczorajszym, natomiast smakowo był bardzo dobry.
Tak samo zresztą jak mój.
Koziemu do jedzenia podany został także nóż ząbkowany - jak przystało do czerwonego mięsa.
Ta restauracja sprawowała się rzeczywiście dużo lepiej niż poprzednia.
Obok nas jakaś rodzina zamówiła przekąskę w postaci grzanek z guacamole.
Podejrzewam, że z jednego awokado zrobili dwie porcje, ponieważ na koszt firmy i my zostaliśmy poczęstowani taką samą przekąska.
Guacamole było bardzo dobrze zmielone - na idealnie kremową konsystencję.
Zgaduję, że restauracja dysponuje wysokiej jakości blenderem.
W tym samym prognozuję sukces ich pinacolady - zarówno lód jak i kokos (którego drobinki rzadko, ale jednak, dało się wyczuć w napoju) były zmielone na kremowa masę.
Kolacja i dwie pinacolady to duża porcja.
Oboje wracaliśmy do pokoju pieszo z pełnymi brzuchami i nadal mokrymi od deszczu kołnierzami.