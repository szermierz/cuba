
\chapter[Smacznego!]{Dzień 2 - Smacznego!}

Wstaliśmy wraz z odgłosem napełniającej się spłuczki.
Woda powróciła!
Oboje szybko wzięliśmy prysznic i stawiliśmy się na śniadanie.
Gonzales przywitał nas talerzem obficie załadowanym pokrojonymi owocami: ananasem, bananami i tym dziwnym czerwonym melonem, którego nazwa nadal nie jest mi znana.
Przygotowane były też grzaneczki, wrzątek, kawa rozpuszczalna wątpliwej jakości (w kwestii wyjaśnienia: na świecie nie istnieje inna kawa rozpuszczalna.
Ale są też i takie, które mi smakują) i cukier.
Na stole stopniowo pojawiały się kolejne talerze, począwszy od solonego masła, poprzez chlebowe zapiekanki z szynką i serem, malutki naleśniczek dla każdego z nas, miodek, ciepłe mleko (idealne do kawy!) i smoothie z czerwonego melona.
Śniadanie było spore, nie podołaliśmy mu w całości. Głównie przez owoce, których było dużo.
Gonzales na koniec próbował dobić nas pytaniem czy chcemy do tego omlet.
Zapłaciliśmy, pożegnaliśmy się z nim i ruszyliśmy do samochodu.
Następnym miejscem na naszym rozkładzie było Viñales.
Jak się okazuje, wynajęte przez nas auto nie jest jakoś super automatyczne.
Brakuje mu między innymi tej wygody, którą jest automatyczne gaszenie świateł.
Odkryliśmy to poprzez objaw rozładowanego akumulatora, automatyczny zamek nie otwierał się, a silnik ani drgnął przy próbie zapłonu.
Z pomocą przyszedł nam Gonzales, który wezwał jakiegoś swojego znajomego mechanika.
Gdy przyjechał, wypiął nasz akumulator, włożył swój, zapalił samochód i operując na otwartym sercu wyjął swój i zamontował z powrotem ten nasz - cały czas na włączonym silniku.
Zalecił nam abyśmy jechali teraz bez świateł, bo zużywają prąd, który musi nam się jeszcze dobrze naładować.
Kubańskie prawo drogowe nie wymaga jazdy ze światłami, mało kto włącza je tu w dzień.
Po raz drugi pożegnaliśmy Gonzalesa i ruszyliśmy realizować plan wycieczki, jaki tego dnia zaczynał się od sklepu z rumem i cygarami.
Odkryliśmy, że Hawana niewiele różni się od Gliwic, przynajmniej z punktu widzenia poszukiwania parkingu.
Wszystkie bowiem w centrum były zajęte lub płatne.
Musieliśmy wykupić sobie godzinę postoju.
Kozi pozostał w samochodzie pilnować zapalonego silnika (kto wie czy znów odpalilibyśmy go na obecnym stanie akumulatora), a ja ruszyłem z lista zakupów przed siebie.
Kiedy wchodziłem do sklepu uliczni obserwatorzy kładli na mnie swoje spojrzenia.
Gdy wychodziłem z zakupami zaczęli mnie pytać, czy wybrałbym się może do fabryki cygar.
Nie mieliśmy jednak na to czasu, pewnie przyjdzie nam jeszcze zostać zaproszonym na taką wycieczkę.
Na pożegnanie spytali mnie tylko skąd jestem.
„Polonia, Fiat” - odpowiedziałem, bo o ile nazywając Polskę po imieniu często trzeba to było powtórzyć, tak nazywając ja Fiatem rozwiewa się wszystkie wątpliwości.
Na Kubie jeździ dużo (około połowa) klasycznych amerykańskich samochodów sprzed rewolucji, które świetnie zachowały się do dzisiaj (jeśli „zachowały” może opisać nadwozie z kitu i zawieszenie wybrednie tolerujące tylko niektóre drogi), rosyjskich Ład, Fiatów identycznych do 126 i 125, ale numerowanych w rozmaity sposób.
Najczęściej pojawiającą się liczbą na małym Fiacie było 450.
Po powrocie do Polski koniecznie to sprawdzę!
Może taki samochód faktycznie figurował też pod tym numerem.
Ostatnią i najmniej liczną grupą są współczesne kompakty, ale bardzo wybrednie dobierane modele.
Jest między innymi jedyny Mercedes: A200 i żadnego innego modelu.
Podobnie widnieją dwa modele Renault: Sandero i drugi o nazwie, której nie zapamiętałem.
Natomiast Fiat 126 lub 126p, czy jakiekolwiek inne oznaczenie identycznie wyglądającego samochodu nazywany jest przez Kubańczyków „polskim fiatem” (tak jak u nas nazywany był małym lub maluchem).
Widziałem nawet dwukrotnie egzemplarz z takim właśnie (”Polski Fiat”) napisem (u nas niemalże każdy 126p miał taki emblemacik z tekstu ułożonego w dwóch odmiennych kolorystycznie liniach).
Kubańczycy oczywiście mówią to po hiszpańsku.
Stad określenie Fiat precyzyjnie opisuje im nasze pochodzenie.
\par Drogi w Hawanie są koszmarne.
Pełne zarówno dziur, jak i rozmaitych wybojów, sięgających średnio 10-15cm, w porywach do 25cm.
Najechanie nawet na przeciętny wybój mogłoby urwać koło.
Nam przynajmniej jeszcze udało się do tego nie doprowadzić.
Poza Hawaną drogi są lepsze - a przynajmniej autostrada, co ciekawe nazywająca się A4, którą jechaliśmy.
Przy drodze jest mnóstwo kubańskich autostopowiczów.
Stoją właściwie w każdym miejscu oferującym odrobinę cienia i machają ręka, żeby ich podwieźć.
Jeden z nich właściwie wbiegł nam prawie pod koła, po czym poprosił żebyśmy go zawieźli do Viñales, gdzie właśnie jechaliśmy. Rzekomo autobus, którym miał jechać, dawno temu nie przyjechał.
Ku mojemu zdziwieniu Kozi się zgodził.
Okazało się, że facet nazywa się Hose (a przynajmniej tak się to wymawia.
Ciężko mi prorokować, jak to napisać, może Jose?) i jest pracownikiem ochrony Kubańskiego Parku Narodowego, wpisanego na listę światowego dziedzictwa kulturowego Unesco.
Puszczał nam swoją ulubioną muzykę z telefonu, tłumaczył zwyczaje jazdy na Kubie, pokazywał zdjęcia żony i córki, opowiadał co możemy zrobić, żeby przenocować w Viñales.
Na Kubie obowiązuje ruch prawostronny, ale jeździ się lewym pasem.
 Prawy jest często dziurawy, więc jeżdżą nim konie i ciężarówki z ogromnymi kołami, którym nie strach taka droga.
Tradycyjnie należy też zatrąbić przy wyprzedzaniu, ale my sobie to odpuścimy, bo ten odgłos co chwile jest strasznie irytujący.
Dowiedzieliśmy się też, że mrugnięcie światłami na autostopowiczów oznacza, że nie weźmiemy ich z powodu pełnego auta lub kończącego się paliwa.
Dziwny sposób...
rzadko widzieliśmy, żeby ktoś zatrzymywał się ich zabrać.
Chyba wydajniej było by sygnalizować, że się ich zabiera, a nie odmawia.
Hose zaproponował też, że pokaże nam drogę przez wspomniany park narodowy, w którym pracuje.
Pojechaliśmy więc, jak się okazało później - dosyć naokoło, przez góry.
Widoki faktycznie były śliczne, ale droga paskudna.
Walka z omijaniem dziur często była bardzo nierówna.
Hose kilkukrotnie prosił, żeby spuścić szyby i pytał miejscowych o drogę, mimo że i tak jechaliśmy na gps’ie.
Pokazał nam, gdzie jest wypożyczalnia łódek, na którą planujemy się jutro wybrać.
Dowiedziałem się też od niego, że ptaki, których jest mnóstwo na Kubie to sępy.
W stadach wirują nad Kubą i szukają pożywienia.
Na koniec zatrzymaliśmy się przy jakimś barze, po środku jakiejś wsi - która okazała się przedmieściami Viñales.
Na tych oto przedmieściach, w przydrożnym barze Hose zaproponował nam sok i jedzenie na jego koszt.
Sokiem było smoothie z mango, za to posiłek był niezwykły.
Składał się z jakiś kopytek z ciasta, którego konsystencja przypominała krzyżówkę kaszy i klusek - w dwóch wariantach - zwykłym i panierowanym.
Wszystko to było dosyć tłuste, a my byliśmy po obfitym śniadaniu, ale jakoś domęczyliśmy te porcje.
Zostaliśmy zaproszeni na ekspresową wycieczkę po plantacji tytoniu.
Tytoń jak się dowiedzieliśmy, był niedawno sadzony i w okolicy grudnia będzie zbierany, a następnie suszony.
Różne części rośliny zbiera się na potrzeby różnych cygar - te które widzieliśmy składały się z liścia wraz z krótkim kawałkiem łodygi.
Na Kubie produkuje się 16 rodzajów cygar, różniących się między innymi aromatem i zawartością nikotyny.
Na koniec wycieczki zostaliśmy poczęstowani cygarem lokalnej produkcji.
Nie mają one filtra, więc pali się je przez koniec zamoczony w miodzie.
Następnie właściciel farmy zaczął na nas naciskać, żebyśmy kupili u niego cygara.
Próbował nam wcisnąć paczki po 25, najlepiej po jednej na każdego z nas.
Koniec końców Kozi zmiękł i kupił od niego 5 cygar, przez co właściciel był niepocieszony, ale dał nam spokój.
Umówiliśmy się na jutro rano na wycieczkę konną.
Usłyszeliśmy od razu, że jego kolega ma pokój do wynajęcia - właśnie tego szukaliśmy.
W nadziei, że nie będzie to wyzysk pojechaliśmy go obejrzeć, ale właściciela nie było - miał się pojawić za parę godzin, z powodu wizyty u dentysty.
\par W samochodzie zostało nam pół baku paliwa.
Ciężko określić spalanie (a ta informacja jest nam potrzebna do oszacowania, kiedy musimy zatankować), bo droga była paskudna, ciągle hamowaliśmy przed dziurami i jechaliśmy w górach.
Hose ostrzegał nas, że na Kubie jest mało stacji benzynowych, ale chyba będziemy dmuchać na zimne i tankować do pełna przy możliwej okazji.
Wyszliśmy na centrum (to duże słowo na jedną ulicę z jednopiętrowymi domkami) Viñales i szukaliśmy baru, w którym popijając alkohol poczekamy na właściciela wynajmowanego pokoju.
Jak się okazało - trafiliśmy na bar z najlepsza pinacoladą jaka piłem!
Za to z dosyć słabym mojito, które zamówił Kozi.
Siedząc napłynęły nam do uszu znajome słowa - obok nas jakieś dwie dziewczyny rozmawiały po polsku!
Kozi wstał kierując się do toalety i widząc, że akurat jadły obiad, rzucił w ich stronę:

\szQuote{„Smacznego!”}
\noindent Oczywiście dziewczyny przekształciły oczy w pięciozłotówki.
Po drodze do pokoju wstąpiliśmy na targowisko upominków, bo jak się okazuje, w Viñales handlują nawet pocztówkami!
Zadałem sprzedawcy trud przeszukania swoich zbiorów, żebym mógł kupić obraz przedstawiający Chevroleta.
I udało się!
Powieszę go sobie w pracy albo nad łóżkiem.
Wróciliśmy po około dwóch godzinach, ale właściciela nie było.
Za to jakiś jego chyba-pracownik kręcący się po posiadłości odpowiedział nam, że za około pół godziny powinien już być - oby!
Gdy przyjechał okazało się, że pokój jest całkiem tani, co zamknęło temat decyzji o noclegu, bo standard pokoju był wysoki, więc tylko tego jednego się obawialiśmy - że będzie drogi.
Realizując plan dzisiejszego dnia wypożyczyliśmy rowery i pojechaliśmy nad miejscowe jezioro Presa el Salto.
Rowery były w słabym stanie, w szczególności hamulce, ale dało się jechać - powoli i ostrożnie.
Na miejscu na zmianę pilnowaliśmy rzeczy i pływaliśmy.
Na zmianę, to znaczy zawsze ktoś czekał na brzegu pilnując naszego dobytku, w tym cennych jak złoto paszportów.
Ten region Kuby wyścielony jest czerwonymi glebami, a góry składają się z jakiegoś żółtego kamienia wymieszanego z szarą masą skalną.
Chciałbym lepiej znać się na składzie gleb, może mógłbym wtedy dokładniej to opisać.
Czerwone ziemie czasami świadczą o obecności żelaza, ale nie jestem przekonany, że tu jest właśnie tak.
Może to jakiś związek fosforu?
Fosfor przyjmuje wiele barw w swoich związkach, przez co nazywa się je kolorami.
Za to fosfor czerwony jest (a przynajmniej tak mi się wydaje) silnie łatwopalny, więc nie będzie raczej składnikiem gleby (bo utleniony zmieniłby kolor).
Dno jeziora było kamieniste do pewnego momentu, dalej było gliniane.
Nie wiem czy gorsze są wbijające się w stopy kamienie, czy mułowe dno.
Wróciliśmy rowerami, umyliśmy się i ruszyliśmy po kolacje.
Pełen entuzjazmu wybrałem miejsce, w którym spożyjemy posiłek, ale wyszło koniec końców na to, że nie był to najlepszy wybór.
Ja zamówiłem szaszłyk wieprzowy, natomiast Kozi stek ze świni.
Mój szaszłyk był taki strasznie zwykły.
Minusem było mięso, było twarde, chyba z jakiegoś starego knura.
Kozi natomiast miał jeszcze gorzej.
Jego „stek” okazał się być schabowym w cebuli, tylko takim bez panierki.
W sumie zaczynam się zastanawiać jak inaczej mógłby wyglądać stek wieprzowy...
Przydarzyło mi się nawet przypadkiem upuścić na ziemie kawałek mięsa przy zdejmowaniu go z szaszłyka.
Chyba miałem szczęście, że nikt tego nie widział, bo bałbym się spalenia na stosie za marnowanie jedzenia - szczególnie mięsa.
Po chwili przypałętał się jeden z tysiąca wygłodniałych psów szwędających się po okolicy.
Wykorzystałem to, ukradkiem podsuwając mu dowód mojej zbrodni.
Zamówiłem sobie losowy drink z karty o nazwie Sangria, Kozi za to wybrał jakąś miodowo-cytrynową mieszankę.
Mój drink był świetny!
Wino z woda gazowana i sokiem porzeczkowym.
Nauczę się taki robić!
Kozi za to trafił na bardzo dobry drink, ale doprawiony alkoholem tak mocno, że ciężko było go pić.
Znam ludzi, którzy nazwali by go średnim - bo chyba zrobionym pół na pół wódki z resztą składników, więc „średnim”.
Hose opowiedział nam dziś żart, że najlepszym sposobem na naukę hiszpańskiego i salsy jest wypicie pięciu mojito.
Chyba nadal nam trochę brakowało, nawet Koziemu.