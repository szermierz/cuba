
\chapter[Yes yes, this is my house]{Dzień 6 - Yes yes, this is my house}

Odgłosy pracującej klimatyzacji Dolores były myląco podobne do przejeżdżającej ulicą ciężarówki. 
Gdyby nie fakt, że sam pokój znajdował się dosyć daleko od ulicy, a w dodatku był od niej oddzielony barierą dźwiękową w formie oddzielnego budynku, mógłbym zastanawiać się czy to na pewno klimatyzacja. 
Ale to była ona.
Po raz pierwszy na Kubie przyszło nam śniadać w towarzystwie.
Przy stole siedziała z nami mówiąca między sobą po francusku para.
Śniadanie było podobne do każdego innego na Kubie. 
Wyróżniało się tym, że zamiast smoothie był (prosto z lodówki!) dzbanek soku z mango, a tost przyszedł w parze z grzaneczką z awokado. 
Na stole były też okrągłe herbatnikowe ciasteczka, ale były tak bardzo średnie, że prawie zapomniałem o tym wspomnieć.
W nocy nasz zaparkowany na ulicy samochód był pilnowany przez lokalny patrol sąsiedzki. 
Wiem o tym dlatego, że przyszło nam za niego zapłacić trzy dolary.
\par Wyjeżdżając z Trinidadu zatrzymaliśmy się w banku, aby wypłacić kolejną porcję gotówki, zgodnie z dziennym limitem transakcji. 
Tym razem udało się bez żadnych przeszkód - zupełnie nie w stylu naszych poprzednich styczności z bankowością na Kubie.
Zapakowaliśmy się w auto i podążając za gps’em ruszyliśmy do Camegüey. 
Jednokierunkowe uliczki Trinidadu w końcu zebrały z nas swoje żniwo i udało nam się w jedną z nich wjechać pod prąd.
Wywołało to burzliwą reakcję otoczenia: przechodnie zaczęli machać na nas rękami i wściekle gwizdać. Gdy zobaczyli, że mimo to próbujemy wyjechać w złym kierunku, dołożyli większych starań i machali dużo bardziej energicznie. Niestety - ich szamańskie zaklęcie nie obróciło nas w dobrą stronę.
Co więcej - uciekając z tej jednokierunkowej uliczki udało nam się wjechać pod prąd w kolejną, co tylko bardziej rozjuszyło tłum. 
Na szczęście po chwili byliśmy już na właściwym torze i cała ta wrzawa ustała. 
To zabawne, że Kubańczycy jeżdżą byle jak, kierunkowskazy traktują jako odświętną ozdobę samochodu, ale dostają białej gorączki widząc kogoś jadącego pod prąd na jednokierunkowej drodze. 
\par Za Trinidadem nastała dla nas spokojna trasa.
Droga oczywiście była dziurawa, ale nie na tyle żeby trzeba było specjalnie zwalniać w obawie o utratę koła. 
Statystycznie rzecz ujmując - im bardziej górzyste są krajobrazy dookoła drogi, tam bardziej jest ona dziurawa.
Droga do Camegüey była śliczna! 
Z trochę większego bliska oglądaliśmy teraz to, co wcześniej widzieliśmy z punktu widokowego.
Mój tata - emerytowany kierownik logistyczny budownictwa drogowego - opowiadał mi kiedyś o profilowaniu zakrętów. 
Zakręt powinien mieć takie nachylenie, aby siła dośrodkowa dociskała pojazd do nawierzchni. 
Dziś komputerowo sterowane maszyny kładą asfalt tworząc precyzyjnie dopasowany kąt profilu nawierzchni, ale kiedyś robione to było na oko i finalna jakość była „taka jak operatorowi wyszło”. 
Kubańskie budownictwo jest jeszcze na tamtym etapie rozwoju, a przynajmniej tak wnioskuję z przechylonych do zewnątrz zakrętów. 
Wspomniane przeze mnie wcześniej jedyne rośliny na Kubie posiadające ogromne rozłożyste liście okazały się być rośliną uprawną. 
Mijaliśmy jej plantację!
Nie dopatrzyłem się za to na nich żadnych owoców, więc albo jeszcze nie owocuje, albo jej zbiory stanowią inną część rośliny albo jest już po zbiorach. 
Ta ostatnia opcja wydaje mi się najbardziej prawdopodobna.
\par Nasz pierwotny plan zakładał spędzenie w Trinidadzie dwóch dni, za to na ostatniej prostej do Hawany przewidywał czternastogodzinną jazdę. 
Postanowiliśmy go trochę przebudować, aby rozbić tak długi trip na dwa mniejsze, kosztem czego ucięliśmy wspomnimy drugi dzień przeznaczony na plażowanie przy Morzu Karaibskim. 
Zamiast niego plażować będziemy w drodze powrotnej - nad Oceanem Atlantyckim.
\par W drodze do Camegüey przejeżdżaliśmy przez kubańską Florydę - miasto o właśnie takiej nazwie.
Architektonicznie było bardzo podobne do Viñales, z tą jedną różnicą, że budynki były w iście hawańskim stanie agonalnym. 
Gdy dojechaliśmy do Camegüey, w pierwszej kolejności udaliśmy się na stację benzynową i na nasze nieszczęście - nie było tu benzyny 94-ro oktanowej. 
W związku z tym, że najbliższa stacja była dopiero na drugim końcu miasta - skierowaliśmy się do centrum w poszukiwaniu noclegu, a tankowanie odłożyliśmy na jutrzejszy poranek - przed wyjazdem. 
Wybraliśmy z mapy jeden o dosyć wysokich ocenach, nazywający się Casa Conchita („casa” oznacza po hiszpańsku „dom”, z kolei hostel wynajmujący turystom pokój to „casa private”). 
Przy wejściu zaczepił nas przechodzień i spytał, czy nie szukamy noclegu. My natomiast chcieliśmy skorzystać z tego wybranego przez nas - w końcu ma wysokie oceny. Dlatego z niechęcią pokazałem mu telefon z mapą i spróbowałem wytłumaczyć, że to jest właśnie miejsce, które chcielibyśmy obejrzeć. Na co on odparł:

\szQuote{Yes yes, this is my house.}
\noindent Zaskoczeni spytaliśmy się, czy możemy tu parkować - na ulicy, na co właściciel zmieszał się i zaoferował pokazać nam drogę na strzeżony parking. 
Wraz z nim pojechaliśmy uliczkami Camegüey na - jak się okazało - całkiem daleko leżący parking.
I gdy już się tam zatrzymaliśmy, wręczono nam pokwitowanie i kazano opłacić opłatę parkingową w wysokości 10 dolarów kubańskich za noc. 
Nie wydało nam się to specjalnie uczciwe.
W dodatku nikt nie spytał nas, jak chcemy dostać się z powrotem do pokoju, a jedynie zapakowano moją walizkę i plecak Koziego na rowerową rykszę. 
Wsiedliśmy do niej i odjechaliśmy...
bez właściciela domu.
Taksówkarz pojechał w drogę powrotną zdecydowanie naokoło. Możliwe, że musiał tak jechać - drobne uliczki były jednokierunkowe. Mam za to pewne obawienia, że w ten sposób chciał zbudować iluzję długości trasy, bo za chwilę kazał nam zapłacić za podwózkę kolejne 10 kubańskich dolarów - kiedy my spodziewaliśmy się, że będzie to w cenie parkingu.
Całe to przedsięwzięcie było już lekkim oszustwem. 
W dodatku staliśmy przy noclegu sporo oddalonym od naszego samochodu, więc jutro rano albo ruszymy pieszo z ciężkimi torbami, albo poprosimy o kolejną przepłaconą taksówkę. 
Odczekaliśmy parę minut, w trakcie których coraz bardziej narastało w nas przekonanie, że nocleg w tym miejscu to nie najlepszy pomysł. W końcu odeszliśmy w kierunku parkingu, poszukać pokoju bliżej.
Z jednej strony tamten właściciel poświecił swój czas, aby zabrać nas na parking - być może on nie widział w tym nic złego.
Za to z drugiej strony nie dotknęliśmy jeszcze klamki pokoju, a już zapłaciliśmy przez niego równoważność jednego noclegu. 
Niemalże przy samym parkingu znaleźliśmy pokój u starszej kobiety, która po przyjęciu nas rozkosznie oznajmiła, że teraz jesteśmy jak rodzina. 
Postanowiliśmy nadać jej imię Conchita, co wydało nam się całkiem zabawne w perspektywie poprzedniego niedoszłego noclegu.
Rozpakowaliśmy się i umyliśmy. 
Nasz plan przewidywał na ten dzień obiad w konkretnym miejscu: restauracji Rancho Luna - miejscu, w którym posiłki są tanie i jednocześnie całkiem dobre - a przynajmniej to wynikało z zebranych przez nas przed wycieczką informacji. 
Udaliśmy się tam i zaczęliśmy przeglądać menu.
Tak jak do tej pory każde było hiszpańsko-angielskie, tak tutaj było czysto hiszpańskie. 
Co więcej, wszystkie dania nazywały się „Table \#n”, gdzie n było liczbą od 1 do 8.
Losując na loterii ja poprosiłem o Table \#4, Kozi natomiast wybrał jedynkę. 
Z bębna maszyny losującej trafiły mi się krewetki, ryż z szynką i szczypiorkiem oraz ugotowany batat. 
Koziemu natomiast danie identyczne oprócz krewetek, które zastąpione były kawałkami wieprzowiny z ogromna ilością tłuszczu z kolagenem. 
Posiłek taki bez wątpienia był niezwykle pożywny, ale i niezbyt atrakcyjny smakowo.
Mój za to był całkiem dobry, ciężko mi było narzekać. 
Każda z naszych porcji kosztowała około 50 peso, czyli 2 dolary.
Bez wątpienia jest to miejsce warte odwiedzenia!
Po obiedzie wybraliśmy się na zwiedzanie Camegüey, trochę w ciemno. 
W samym centrum miasta znajdował się park, w sercu którego miała być jaskinia.
Atrakcja ta wydawała nam się tak egzotyczna, że bez wahania zaczęliśmy od niej. 
Park faktycznie był w miejscu wskazanym na mapie, co ciekawe był to park z drzew iglastych!
Wprawdzie gatunku, którego nigdy wcześniej nie widziałem, mającego w sobie coś z wierzby płaczącej. 
Po powrocie do kraju koniecznie dowiem się jak się nazywa!
W samym środku parku znajdowało się coś, co z daleka przypominało wejście do groty osadzonej w górze.
Tylko bez góry. 
Gdy podeszliśmy bliżej, wejście to okazało się poprawioną betonem konstrukcją, symbolicznie będącą jaskinią, wyścieloną w środku ozdobami. Przypominało trochę naszą szopkę bożonarodzeniową.
\par Następnym punktem wycieczki był skwer leśny pośrodku osiedli. 
Podejrzewałem, że był to jakiś zagajnik lub inne miejsce, pozwalające nam usiąść w cieniu i odpocząć od upału.
W drodze tam mijaliśmy prawdziwy stadion, na którym właśnie odbywał się mecz! 
Do tej pory nie widziałem nigdzie na Kubie żadnych obiektów sportowych, jeśli nie liczyć boiska 5 na 5 metrów wyścielonego betonem, na którym dzieci podrzucały piłkę. 
Pod stadionem zaparkowany był autokar ze wsadzoną za szybę tabliczka opisaną „béisbol equipo”, co pozwalało zgadywać jakiego sportu mecz właśnie trwa.
\par Leśny skwer, do którego dotarliśmy okazał się nie istnieć, więc skierowaliśmy się do ostatniego punktu harmonogramu - Parku Japońskiego. 
Przez około godzinę maszerowaliśmy na przedmieścia Camegüey, a gdy już tam dotarliśmy, naszym oczom ukazał się park imienia Camilo Cienfuegos. 
Park ten nie był japoński, był za to wesołym miasteczkiem przypominającym chorzowskie wesołe miasteczko sprzed 20 lat. 
Mogło to być spowodowane sobotnim popołudniem, ale park ten był w tym momencie nieczynny, a atrakcje pozamykane. 
Ochrona obiektu nie chciała nas w ogóle wpuścić, ale nie potrafili mówić po angielsku, a my po hiszpańsku, więc po chwili komunikacyjnych zmagań odpuścili nam i spacerkiem ruszyliśmy w głąb parku. 
Pełno tam było atrakcji w stylu klasycznych elektrycznych samochodzików do zderzania, kręcących się karuzel, a nawet jeden gabinet luster.
Wszystko oczywiście pozamykane. 
Odpoczęliśmy chwile w cieniu i ruszyliśmy w drogę powrotną do pokoju.
Wracając wpadliśmy na pomysł, że moglibyśmy kupić sobie jakiś orzeźwiający złocisty napój na drogę. 
Rozglądając się za sklepem udało nam się znaleźć prawdziwą perłę kubańskiego handlu: supermarket.
Weszliśmy do środka, wzięliśmy koszyk z kategorii tych mniejszych - trzymanych w ręku. 
Wybraliśmy taki, ponieważ żadnego innego rodzaju nie było. 
Wystrojem sklep przypominał samoobsługowy magazyn Ikei, natomiast na półkach faktycznie znajdowały się produkty różnych kategorii, co dotąd było niespotykane na Kubie. 
Z relacji hawańskiego przewodnika wyniosłem, że i tam znajdują się takie sklepy - ale na całe miasto są tylko dwa i w dodatku niezbyt dobrze zaopatrzone. 
My rozbiliśmy się tu o podobny problem: o ile było tu kilka palet rumu różnego rodzaju, tak nie było ani jednej puszki piwa. 
Zrezygnowani wzięliśmy po puszce jakiegoś bezalkoholowego słodzonego napoju i udaliśmy się do kas, a były aż dwie. 
Wychodząc ze sklepu musieliśmy jeszcze pokazać rachunek za wszystko co kupiliśmy, a na koniec kontroli nasz rachunek został przetargany. 
Sam napój był ciepły, co w połączeniu z tym, że był przesłodzony, sumarycznie niezbyt gasiło pragnienie. 
Słońce powoli zachodziło, więc korzystając z nareszcie niższej temperatury wybraliśmy się na drinka do baru w pobliżu noclegu.
Trafiliśmy na tematyczną miejscówkę zewsząd obklejoną plakatami The Beatles. 
Przy wejściu przywitały nas posagi Paul’a, George’a, John’a i Ringo.
Ku naszemu niezadowoleniu i tu nie było piwa.
Zamówiliśmy za to pinacoladę i mojito, które okazało się bardzo dobre! 
Z przyjemnością wypiliśmy jeszcze po dwa takie i wróciliśmy do pokoju.
Nie było jeszcze specjalnie późno, bo powoli dochodziła godzina 21-sza, ale zmęczenie bezlitośnie pociągnęło mnie do snu.
