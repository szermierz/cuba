\chapter[Wstęp]{Dzień 0 - Wstęp}
Żaden ze mnie pisarz.
Mimo to mam nadzieję drogi czytelniku, że przypadnie Ci do gustu ten opis podróży przez Kubę, która odbyła się między pierwszym, a trzynastym września 2019 roku.
Cała opisana tu historia wydarzyła się naprawdę (choć nie spodziewaj się zapierających dech w piersiach scen), a dziennik ten jest jej wiernym odzwierciedleniem, wzbogaconym o elementy kultury, historii, ekonomii, polityki i kulinariów Kuby.
Razem z moim przyjacielem Karolem Kozim Kozok („Kozi” wymawia się z twardym ’z’, nie jako ’ź’) zdecydowaliśmy się w ciągu dwóch tygodni pojechać w trasę dookoła Kuby i posmakować jej na możliwie najwięcej sposobów.
Plan naszej wycieczki został opracowany przez Koziego w oparciu o reportaż innej dwójki podróżników i zmodyfikowany na nasze potrzeby.
Przede wszystkim, aby uniezależnić nas od poruszania się po Kubie i tym samym zdążyć w dwa tygodnie, zdecydowaliśmy się wynająć samochód. Stał się on naszym trzecim kompanem, właściwie już zaraz po wyjściu z lotniska, bo właśnie tam znajdowała się wypożyczalnia samochodów.
Nie marnując ani chwili pojechaliśmy za nawigacją do centrum Hawany, gdzie mieliśmy zarezerwowane pierwsze dwa dni noclegu.
Tam odebraliśmy klucze do pokoju i zmęczeni położyliśmy się spać.