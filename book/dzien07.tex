
\chapter[Najwyższy czas wyjechać z tego miasta]{Dzień 7 - Najwyższy czas wyjechać z tego miasta}

Klimatyzacja w pokojach, w których nocowaliśmy pracowała dosyć nierówno.
Wieczorem, kiedy wracaliśmy, w pokoju było rozkosznie chłodno.
Braliśmy prysznic i przy przejściu z łazienki do pokoju, nadal było przyjemnie.
Nad ranem temperatura też była w sam raz.
W środku nocy z kolei pokój magicznie znajdował się przez parę godzin na biegunie.
Niezależnie od ustawień klimatyzacji.
W okolicy godziny 2giej w nocy robiło się strasznie zimno.
Tej nocy nawet dość zimnolubny Kozi owinął się ciasno w narzutę i poszukiwał ciepła.
Chyba jedynym rozwiązaniem będzie wyłączać klimatyzacje, gdy obudzimy się zmarznięci.
\par Śniadanie u Conchity oprócz zwykłych elementów tego posiłku na Kubie, zawierało rozcieńczony sok z mango, a zamiast szynki wyłożone były usmażone plasterki parówki.
Podejrzewam, że tej samej, która stanowiła główny składnik hawańskich hot dogów w żółtej bułce z okienka.
Smakowała jak każdy inny przetwór mięsny zawierający dokładnie każdą część świni, która da się zmielić.
Czyli każdą.
Jak co rano zapłaciliśmy i pożegnaliśmy się.
Ruszyliśmy z bagażem na parking, który leżał około 50m w linii prostej od nas.
Odebraliśmy auto i pojechaliśmy zatankować, co było ostatnim punktem planu w Camegüey.
Gdy Kozi tankował, ja uważnie obserwowałem mężczyznę, który podjechał rowerem pod dystrybutor, wyciągnął z plecaka plastikową butelkę, wsunął w nią dyszę i nacisnął spust.
Średnica dyszy była minimalnie węższa od średnicy otworu butelki, a dystrybutor wstrzykiwał paliwo w tempie niemal litra na sekundę.
W rezultacie opryskał on paliwem cały dystrybutor oraz siebie - głównie po twarzy.
Niestrudzenie wyciągnął z plecaka drugą butelkę i tym razem ostrożniej lejąc, napełnił ją do połowy.
Następnie i pierwszą napełnił w połowie, po czym odjechał bez płacenia.
A przynajmniej tak mi się wydawało, bo później Kozi rozwiał moje wątpliwości informując, że na Kubie najpierw się płaci, a dopiero potem operator odblokowuje dystrybutor.
Dzięki czemu taka operacja nie jest możliwa.
\par Odjechaliśmy ze stacji i skierowaliśmy się do Santiago de Cuba.
Zwrot „de Cuba” oznacza „z Kuby” lub „kubański”.
Na drogowskazach kilkukrotnie widziałem krótszą wersję: „Sntg de Cuba”.
Budownictwo na Kubie jest dosyć konsekwentne - domy wyglądają bardzo podobnie, występują w kilku powtarzających się schematach kolorów.
Na Kubie jest jedna konstrukcja wieży ciśnień, którą widziałem dziesiątki razy.
Różnica między kubańskimi miastami leży raczej w tym, jak bardzo budynki nadają się do remontu.
Trinidad wyróżniał się sklepieniami domostw pokrytymi dachówką, 
Hawana natomiast posiadała dachy z rozpadających się najwyższych kondygnacji.
\par Podczas trwającej około 5 godzin trasy, mijaliśmy Las Tunas.
Śliczne miasteczko o niebywałej jak na Kubę cesze: zakazie używania klaksonu.
Aż przykro mi było, że nie znajdowało się ono na naszym rozkładzie zwiedzania.
Oczywiście mogliśmy zmodyfikować nasz plan, ale byłoby to nierozsądne.
Zależało nam na tym, aby dojechać do wschodniego wybrzeża, możliwie blisko przylądka.
Jeśli chcielibyśmy zachować ten cel i jednocześnie zatrzymać się w Las Tunas, musielibyśmy wyciąć jeden z postojów w drodze powrotnej do Hawany, a to oznaczałoby powstanie ponad 10cio godzinnej jazdy.
A drogi na Kubie wymagają ciągłego skupienia, więc tak długa jazda była by zarówno męcząca, jak i niebezpieczna.
W Santiago de Cuba upatrzyliśmy (na mapie offline) nocleg możliwie blisko wody.
Zajechaliśmy na miejsce i odnaleźliśmy adres hostelu.
Ganek domostwa okrążony był kratą i nigdzie w zasięgu rąk nie było dzwonka.
To dosyć ciekawe - jak w takim razie turyści mają zapukać do drzwi, skoro nie da się ich dosięgnąć?
Dokładniejsze oględziny krat wykazały, że furtka wprawdzie posiadała kłódkę, natomiast nie była zamknięta.
Moglibyśmy ją zdjąć, a następnie wejść i zapukać.
Wydawało nam się to już delikatnie podchodzące pod włamanie, więc zdecydowaliśmy ściągnąć ręką przez kraty i zapukać w drewniane okno.
Tak też zrobiłem i po dłuższej chwili drzwi otworzyły się i naszym oczom ukazała się właścicielka domu.
Poprosiła nas, abyśmy chwile poczekali, a my wykorzystując ten moment podjęliśmy decyzje, by nadać jej imię Helga de Santiago.
Gdy wróciła, pokój był posprzątany i gotowy, aby nas przyjąć.
\par Zostawiliśmy bagaże i ruszyliśmy w miasto, w pierwszej kolejności oczywiście w kierunku zatoki o kaskadowej nazwie: Bahía de Santiago de Cuba.
W Santiago de Cuba nie ma plaży, woda obija się jedynie o betonowe nadbrzeże.
Nad nim z kolei leży przystań z na stałe zacumowanymi okrętami pełniącymi funkcje restauracji.
Obok nich jest słynny trzymetrowy napis Cuba, z którym zrobiliśmy sobie pamiątkowe zdjęcia.
Gdy próbowaliśmy się ustawić, zaczepił nas jakiś facet oferując nam usługi przewodnika.
Odmówiliśmy, bo mieliśmy już przygotowany plan zwiedzania.
Ruszyliśmy do centrum miasta w poszukiwaniu restauracji.
Udało nam się znaleźć całkiem dobrze wyglądająca knajpkę, w której ja zamówiłem kurczaka w sosie barbecue, natomiast Kozi poprosił o kurczaka w czosnku.
„Sos barbecue” okazał się był czerwonym słodkim sosem o aromacie Maggie.
Do potrawy podana była fasolka szparagowa oraz łyżka żółtych ziemniaków.
Bardzo chciałem spytać kelnera co to za gatunek ziemniaka, ale usilnie nas ignorował i w ten sposób nigdy się tego nie dowiedziałem.
Po posiłku skierowaliśmy się do pierwszej lokalnej atrakcji, jaką był park Plaza de Marte, ozdobiony licznymi pomnikami.
Ten największy, w centrum parku, był oczywiście pomnikiem José Martí, który jest najczęściej uwiecznianym na pomnikach Kubańczykiem.
Drogi w Santiago do Cuba są dosyć strome.
Jedna przecznica o długości 25m potrafi wznosić się na 10m wysokości. Maszerując ulicami miasta często przychodziło nam wspinać się pod górę, żeby za chwilę schodzić w dół. I zza horyzontu wyłaniał się identyczny geometrycznie odcinek.
\par Drugi odwiedziny przez nas park - Parque Histórico Abel Santamaría - przywitał nas ogromną (wyłączoną) fontanną, na szczycie której znajdował się betonowy sześcian, na którego bocznych ścianach wyryte były wklęsłe twarze bohaterów Kuby. Nie zabrakło oczywiście José Martí, ale poza nim nie rozpoznałem żadnego.
Usiedliśmy na ławce i odpoczęliśmy dobre kilkadziesiąt minut i gdy tylko odzyskaliśmy trochę sił, skierowaliśmy się do następnego punktu programu: na punkt widokowy.
Dotarcie tam wymagało wejścia po wysokich betonowych schodach, w trakcie czego wesoło nuciliśmy sobie Eye of the Tiger.
Na szczycie jednak okazało się, że widok na Santiago de Cuba nie był niestety powalający.
Sam punkt widokowy był kawałkiem osiedla, które zawierało ichniejszy plac zabaw - dzieci kopały piłkę, gdzieś leciała muzyka.
Odsapnęliśmy chwilę po wspinaczce, właściwie to dłuższą chwilę, bo upał utrudniał nam zebranie sił, po czym wstaliśmy i zaczęliśmy schodzić w dół - w kierunku centrum.
Nasz nocleg znajdował się w pobliżu zatoki, a w niej stała wiecznie zakotwiczona barka, której bar obraliśmy sobie za następny przystanek.
Santiago de Cuba jest drugim największym miastem na Kubie - zaraz po Hawanie. Odczuliśmy to w nogach - przemieszczanie się między jego atrakcjami w upalny dzień było wycieńczające, szczególnie pokonywanie stromych przecznic.
\par Kiedy już usiedliśmy przy stoliku na stalowej barce, zgodnie postanowiliśmy nie wyruszać już dziś poza okolice noclegu.
Obok nas rozgościła się cała kubańska rodzina, która w dodatku przyniosła z domu swoje piwo w plastikowych butelkach. Wszędzie był straszny hałas, z głośników grała jakaś muzyka, której nawet nie słyszałem, bo z tej odległości basy przyćmiewały całą resztę spektrum.
Nie wytrzymaliśmy w tych warunkach za długo i gdy tylko dopiliśmy piwo, zapłaciliśmy.
Obsługa barki przetrzymała nas jeszcze 10 minut ociągając się z wydaniem reszty.
\par Wciąż odczuwając pragnienie ruszyliśmy szukać jakiegoś miejsca, w którym moglibyśmy zamówić drinka.
Udało nam się znaleźć dosyć elegancką restaurację, w której usiedliśmy pod gołym niebem.
Obsługa wspomniała nam, że do północy odbywać się tu będą tańce, ale poza nami nikogo w niej nie było, więc wydaje mi się, że obawa ta była trochę na wyrost.
Zamówiliśmy pinacoladę i mojito, ale po chwili kelner przyniósł nam dwie pinacolady.
Co ciekawe - jak wchodziliśmy, to właśnie takie dwa drinki stały już na barze.
Jestem niemalże pewien, że po prostu opchnęli je nam, licząc chyba, że nie będziemy protestować i Kozi zrezygnuje ze swojego mojito.
I faktycznie odpowiedziałem, że to nie problem. W końcu i tak będę chciał zamówić drugą, za to Kozi poprosił o podanie faktycznie zamówionego drinka.
Oboje byliśmy już zmęczeni tym miejscem.
Począwszy od lokalnego przewodnika przy napisie „Cuba”, poprzez nieustająco klejących się do nas żebraków, grupkę dzieci która lazła za nami od punktu widokowego aż po barkę, taksówkarzy którzy oferowali nam podwózkę, a po odmowie oferowali sprzedaż bilety autobusowego, a po kolejnej odmowie kartę dostępu do Internetu, po ciągłe gwizdy i pytania z tłumu „hello, where are you from”.
Całkowicie rozumiem, że Ci ludzie żyją w biedzie i dla nich każdy 1 na 100 turystów, który zatrzymuje się podzielić swoim portfelem jest na wagę złota.
Ale na dłuższą metę zaczęło nam to działać na nerwy.
Wisienką na torcie były te właśnie zamówione przez nas drinki.
Najpierw wciśnięto nam dwie pinacolady, gotowe jeszcze zanim przeszliśmy przez próg, a potem gdy Kozi poprosił o drugie mojito, zamiast drinka otrzymał swoją poprzednia szklankę z miętą uzupełnioną do pełna wodą gazowaną. To była już kpina.

\szQuote{„Najwyższy czas wyjechać z tego miasta”}
\noindent - powiedział Kozi.
 
Zapłaciliśmy nie zostawiając krztyny napiwku i zrezygnowani wróciliśmy do pokoju.
Gdy już położyliśmy się do łóżek okazało się, że nasz świetny pomysł z noclegiem przy wodzie zaowocował nam nocną imprezą pod samymi oknami aż do 3ciej nad ranem.