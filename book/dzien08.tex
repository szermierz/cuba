
\chapter[Bunkrów nie ma, ale też jest zajebiście!]{Dzień 8 - Bunkrów nie ma, ale też jest zajebiście!}

Obudziliśmy się ledwie przytomni.
Spaliśmy może po 4 godziny.
Już nie śpiąc, leżałem zniesmaczony tym, jak wyglądała noc prawie do samego śniadania.
Ostatnie 10 minut przed umówioną godziną posiłku wstałem, umyłem twarz mydłem i razem z Kozim wyszliśmy z pokoju na piętro niżej, gdzie czekał już zastawiony stół.
Chciałem powstrzymać negatywne nastawienie, ale przygotowane nam śniadanie dotrzymało kroku całemu Santiago do Cuba i było najgorsze jak do tej pory.
Omlet pływał w tłuszczu, kawałki melona smakowały jak wyciągnięte z puszki, a jedyną formą mięsa były pieczone kawałki schabu, niestety również strasznie tłuste, chyba smażone razem z jajkami.
Podziękowaliśmy i odjechaliśmy.
Ostatnim punktem do odhaczenia było napełnienie baku do pełna.
Odwiedziliśmy dwie stacje benzynowe, ale żadna nie oferowała benzyny 94-ro oktanowej.
Opuściliśmy więc miasto i udaliśmy się w trasę, nawigację ustawiając na kolejne stacje przy drodze.
Poczuliśmy się trochę zagrożeni, gdy poziom paliwa z każdym kilometrem malał, a stacja w Alto Songo również nie pozwoliła nam napełnić baku.
Na całe szczęście kawałek dalej, w Le Maya w końcu nam się udało i pewni siebie skierowaliśmy nawigację na Baracoa.
Im dalej na wschód, tym klimat robił się coraz bardziej suchy.
Gleby traciły swój pomarańczowy kolor, a stawały się bardziej piaskowe.
W pewnym momencie po obu stronach jezdni połowę roślinności stanowiły kaktusy!
To niesamowite jak kilkaset kilometrów może odmienić krajobraz.
Ostatni raz widziałem coś takiego w Disneylandzie.
\par W trakcie jazdy do Baracoa trafiła nam się pierwsza prawdziwa kontrola policyjna.
Kontrole drogówki na Kubie najczęściej występują w specjalnie wyznaczonych miejscach: „Punto de Control” - ale bywają i wyjątki.
Funkcjonariusze poprosili o nasze dokumenty, ale gdy zobaczyli tylko, że wyciągamy z torby paszporty, powiedzieli, że możemy jechać dalej.
Zacząłem się zastanawiać, czy te kontrole dotykają losowo zatrzymywane samochody, czy może też punkt kontrolny w tej okolicy po prostu nie zanotował żadnego kierowcy od paru godzin i sprawdził nas z nudów.
\par Po przejechaniu paru kilometrów droga zbliżyła się do krawędzi lądu i tym samym sunęliśmy przy Morzu Karaibskim.
Ucieszeni jak dzieci postanowiliśmy zatrzymać się tu i popływać.
Wysiedliśmy z samochodu i rozprostowaliśmy nogi.
Poczułem się jak filmowy Laska z Polski.

\szQuote{Bunkrów nie ma, ale też jest zajebiście!}
\noindent Tafla morza była intensywnie niebieska.
Zostawiliśmy większość rzeczy w samochodzie i zeszliśmy w dół kamiennej plaży. 
Mimo tego, że najpierw była moja kolej na pilnowanie, czas mijał bardzo przyjemnie.
Rozkoszny widok Morza Karaibskiego rozciągającego się u moich stóp był jednym z tych miłych akcentów, których brakło w Santiago de Cuba.
Kamienie na plaży były ciekawie podziurawione.
Niektóre miały łagodną powierzchnię, ale były też takie o rozmaitych...
żyłach?
Ale i dziurkach, wzory były różniste.
Gdy Kozi wrócił na brzeg, podekscytowany zdjąłem buty i powolutku wszedłem do wody po kamiennym dnie.
Woda była idealnie przejrzysta, dzięki czemu bez problemu stawiałem kolejne kroki.
Nawet będąc już daleko po pas w wodzie, nadal dokładnie widziałem dno, było jedynie rozmazane przez fale większe od tych przy samym brzegu.
Pływając z wynurzoną głowa kilkukrotnie rozbijane o mnie fale chlapnęły mi na usta i nie zdziwię chyba tym nikogo, że woda była słona.
Po pewnym czasie wyszedłem na brzeg i oboje odczekaliśmy trochę czasu, aby wyschnąć.
Następnie przebraliśmy się, zapakowaliśmy w auto i ruszyliśmy w dalszą trasę.
Kawałek dalej znowu zostaliśmy zatrzymani przez policję.
Tym razem nie zdążyliśmy nawet wyciągnąć paszportów - gdy policja usłyszała, że mówimy jedynie po angielsku puściła nas dalej.
Może ktoś z podobnym do naszego autem jest tu poszukiwany?
Kto wie, mam nadzieję, że nie przysporzy nam to problemów.
\par Droga do Baracoa odbiła na północ i wkroczyła pomiędzy góry.
Zakręty zaczęły się robić coraz ciaśniejsze i bardziej kręte.
Nawierzchnia jezdni była w dobrym stanie, ale studzienki ściekowe wydrążone tuż przy drodze były spore i bardzo głębokie, jednocześnie mało która z nich była czymkolwiek przykryta.
Zresztą podejrzewam, że najechanie na tę zakrytą kratami niczym się nie różni od najechania na całkiem gołą - koło po prostu całkowicie wpadnie w dół popychając przed sobą kratę lub nie.
Trasa taka towarzyszyła nam praktycznie do samego Baracoa.
Jutro będziemy tędy wracać, bo jest to jedyna cywilizowana droga prowadząca w te rejony.
Moglibyśmy odbić z Baracoa bezpośrednio na zachód, ale mapy ostrzegają nas, że niejednokrotnie zabraknie nam wtedy asfaltu.
\par Miasto okazało się całkiem przytulne, a przede wszystkim spokojne.
Udało nam się nawet znaleźć nocleg w całkiem ładnym pokoju za pierwszym podejściem.
Właścicielkę pokoju nazwaliśmy Mariją.
Pokój znajdował się na piętrze i na dwóch z jego ścian była kaskada małych okienek.
Każdy z nich miało jedne drewniane drzwiczki, które otwierane były w sposób podobny do zaworów wodnych.
Obracały się względem osi przechodzącej przez środek drzwiczek tak, że ich pozycja zamknięta zamykała okno, a otwarta skierowana była równolegle do przepływu światła i powietrza.
To rozwiązanie jest bardzo proste - dzięki czemu takie okna są tanie i można ich zamontować więcej na ścianie - jak i dosyć wydajne, bo w ten sposób więcej okien oznacza większy przepływ powietrza - stąd znajdowały się one aż na dwóch ścianach pokoju.
Jest tylko jeden mankament - okna takie nie są właściwie wcale dźwiękoszczelne.
\par Wyruszyliśmy na obiad w kierunku pizzerii, zgodnie z moim życzeniem.
Jedliśmy na Kubie różne rzeczy, ale pizzy jeszcze nie.
Ciekawe jak smakuje kubańska adaptacja włoskiego jedzenia!
I jak się niestety okazało, nie przyszło nam jej spróbować, bo w miejscu, gdzie na mapie widniała pizzeria, w Baracoa znajdował się zamknięty zewsząd zwykły dom.
Zaproponowałem poszukać kolejnej, ale Kozi wypatrzył już wysoko ocenioną restaurację na mapie, więc udaliśmy się właśnie tam - do El Buen Sabor.
W restauracji przywitał nas zaskakująco dobrze mówiący po angielsku mężczyzna.
Na wstępie poinformował nas, że dania w karcie zawierają także przystawki, których pełną listę wymienił.
Ja zamówiłem kurczaka w sosie kokosowym, Kozi za to postanowił po raz kolejny zmierzyć się ze stekiem wieprzowym.
Najpierw podano nam pyszne czipsy bananowe, następnie coś nazwanego przez kelnera rybimi krokietami.
Były to kulki z ciasta o smaku ryby, smażone na głębokim oleju.
Do dania głównego dostaliśmy dużą miskę brązowego ryżu smażonego z fasolą.
Mój kurczak był świetny!
Stek Koziego również, ale w pojedynku sam na sam nie miał szans z kurczakiem.
Po posiłku kelner spytał czy wszystko było w porządku, a następnie ile dni będziemy w Baracoa.
Gdy usłyszał, że pozostajemy tu tylko na jeden dzień i rano planujemy wyjechać, uprzedził nas, że w Baracoa jest wiele do zobaczenia i żeby wystarczyło czasu na wszystko musielibyśmy zostać tu około trzech dni.
Dodał też, że jest też lokalnym przewodnikiem i pokazał nam na mapie gdzie jest plantacja kakao oraz gdzie będziemy mogli wypożyczyć łódkę i płynąc podziwiać dziką naturę.
Coś czuję, że liczył, że wynajmiemy go na przewodnika, ale mieliśmy wtedy przed oczami perspektywę zobaczenia wielu rzeczy i niewiele czasu na ich oglądanie, więc nie chcieliśmy spowalniać wycieczki - a tak by właśnie było z przewodnikiem.
Podziękowaliśmy i zapłaciliśmy, po czym szybko wsiedliśmy w samochód i pojechaliśmy na wskazaną plantacje kakao.
Mieliśmy tam szukać kobiety o imieniu Stokrotka (nie po polsku oczywiście), która będzie mogła przedstawić nam proces powstawania wytworów czekoladowych.
\par Stokrotka okazała się być kobietą w podeszłym wieku, która niezbyt dobrze mówiła po angielsku. 
Za to zaproponowała nam, że o ile - jak sama powiedziała - to ona jest królową kakao, tak nas może oprowadzić jej córka - księżniczka kakao.
Zostaliśmy poprowadzeni na pola kakaowe.
Był to moment, w którym jedna z zagadek Kuby, która trapiła mnie od dawna została rozwiązana.
Na polach znajdowały się trzy rodzaje roślin: kawowiec, którego mieliśmy okazje widzieć już wcześniej, kakaowiec - podobne do niego drzewko, może trochę większe, z mnóstwem pączkujących kwiatów lub całymi już owocami kakaowca, oraz trzeci rodzaj: średniej wielkości palma z ogromnymi rozłożystymi liśćmi.
Był to bananowiec!
Możliwe, że plantacja bananowca którą wcześniej widziałem przy drodze była już po zbiorach lub zwyczajnie jadąc dosyć szybko samochodem nie zauważyłem zielonych jeszcze kiści bananów ukrytych pod tymi ogromnymi liśćmi.
Bananowce sadzi się na plantacjach kakao, aby chronić drzewa kakaowe przed słońcem.
Każde drzewo kakaowca daje w ciągu roku między 100 a 150 owocami.
Owoce zbiera się, gdy robią się brązowe lub purpurowe, wtedy po odcięciu kawałka skóry widać, że w środku jest żółty - gotowy do zbioru.
Farma wysyła większość owoców do fabryki, ale pewną część zostawia i produkuje wyroby kakaowe lokalnie.
W lokalnej produkcji owoce rozcina się i wyciąga ze środka nasiona razem z białą kleistą otoczką o smaku podobnym do liczi - subtelnie kwaśnym i słodkim jednocześnie.
Gromadzi się je w koszykach i pod przykryciem z liści bananowca zostawia do sfermentowania.
Gdy są już gotowe, odsącza się je z powstałego płynu - likieru kakaowego i suszy, a następnie mieli w młynkach ręcznych.
Powstaje w ten sposób ciapka, która podgotowana rozpada się na masło kakaowe i pozostałości.
Samo masło jest składnikiem produkcji czekolady, ale funkcjonuje też jako kosmetyk nawilżający skórę.
Lokalna produkcja wyrobów czekoladowych wyrabia też i samo masło, ale większość powstałej po zmieleniu ciapki w całości formowana jest w kulki o promieniu około 7 cm i suszona, następnie kulki takie tarte są na proszek na ręcznych tarkach.
Z proszku powstaje tu gorzka czekolada lub jedna z dwóch mieszanek czekolad słodkich: z proszkiem bananowym i cynamonem, oraz proszkiem bananowym i miodem.
Fabryki czekolady produkują z owoców półprodukty: znane nam kakao naturalne i masło kakaowe.
Lokalna produkcja do wyrobów czekolady korzystała ze składników bez wydzielonego masła, fabryka zaś wysyła je dalej, gdzie ponownie są łączone w odpowiednich proporcjach i powstaje produkt bardziej jakościowy.
Natomiast moim zdaniem - czekoladkom z farmy niczego nie brakowało!
\par Po wycieczce pojechaliśmy do Yumuri, wioski nieopodal Baracoa.
Przejechaliśmy przez most, u stóp którego leżała przystań, wróciliśmy się więc zaparkować i poszukać wypożyczalni łódek.
Zanim dojechaliśmy do końca mostu na drogę wybiegł nam jakiś mężczyzna i usilnie pukał w szybę, aż opuściliśmy ją na dół.
Powiedział nam, że jest tu przewodnikiem i może zaoferować nam wycieczkę wodną.
Oczywiście gdy tylko dojechaliśmy do wskazanego przez niego parkingu okazał się on płatny.
Wydaje mi się jednak, że jest tak samo płatny jak niektóre darmowe parkingi w samym centrum Katowic.
Jeśli zapłacisz menelowi za postój, to nie porysuje Ci auta.
Ruszyliśmy pieszo za naszym przewodnikiem, poczekaliśmy na molo i po chwili przypłynęła łódź wraz z wioślarzem.
Wsiedliśmy i popłynęliśmy wzdłuż rzeki, która płynęła przez swojego rodzaju kanion, a przynajmniej z obu stron otoczona była stromymi górami.
Spodziewałem się, że popłyniemy dalej, ale wioślarz skierował łódź do pobliskiego brzegu i wraz z przewodnikiem wysiedliśmy na ląd.
Rozpoczęło to naszą pieszą wycieczkę.
Przewodnik wskazał nam szałwię rosnącą wzdłuż rzeki i dodał, że znajduje się tu jej plantacja.
Wydaje mi się, że rośnie ona tylko dziko przy rzece, ale może faktycznie gdzieś w pobliżu jest jakaś.
Chyba nigdy się tego nie dowiem.
Wskazał nam też krzew mango, kilka innych roślin.
Na kamienistej plaży znajdowała się dobrze zachowana skorupa kraba - widocznie niedawno zrzucona.
Czułem się trochę jak ofiary przewodników po Taj Mahal w Slumdogu.
Po chwili przewodnik dodał, że dalsza część wycieczki jest dodatkowo płatna i zawiera zwiedzanie plantacji kakao i wioski indiańskiej.
Jako że już to widzieliśmy, a wspomniana przez niego wioska indiańska zapewne byłaby jednym namiotem, odmówiliśmy.
Wystarczało nam już wyzysku jak do tej pory.
Słysząc naszą rezygnacje przewodnik zaproponował zmniejszyć cenę z 25 do 20 dolarów na osobę, co i tak było ogromnym wydatkiem.
No i nie chodziło tu o cenę.
Zaczęliśmy się zatem wracać, a przewodnik opowiadał nam, że jego dom został zniszczony przez huragan 5 lat temu i razem z żoną i dwójką dzieci żyją w nędzy.
Zrobiło mi się przykro i był to ten gorzki rodzaj przykrości, pomieszany z tym, że wymuszono na nas płatny parking i że całe to udawane zwiedzanie i tak kosztuje 10 dolarów.
Z czego jak się dowiedzieliśmy 2 trafią do niego, a 8 do jego szefa.
\par Wróciliśmy łódką do przystani, a następnie pieszo do samochodu.
Otworzyliśmy mapę i zaczęliśmy szukać najdalej wysuniętego na wschód punktu, jaki jesteśmy w stanie osiągnąć samochodem.
Byliśmy w końcu blisko wschodniego przylądka Kuby!
I jak się okazało, moglibyśmy dojechać na sam geograficzny przylądek, ale zajęłoby to 3 godziny (w jedna stronę), bo trzeba by tam jechać całkowicie dookoła, z powrotem przez góry.
Pojechaliśmy zatem kawałek dalej, gdzie znajdował się punkt widokowy.
Był to szczyt klifu z pięknym widokiem na Ocean Atlantycki.
Znajdowała się tam też budka z napisem „bar”, więc uradowani poszliśmy poszukując czegoś na schłodzenie.
Niestety napis był bardzo mylący, bowiem w środku nawet stała jakaś lodówka, a nad nią butelki z rumem, ale kupić tam można było jedynie cygara i papierosy.
Wyszliśmy z niej i zrobiliśmy ostatnią rzecz, jaka nam tam pozostała - rozkoszowaliśmy się widokiem rozciągającej się u stóp klifu wody.
Gdy już nacieszyliśmy oczy, wróciliśmy do Baracoa.
Tam zaparkowaliśmy pod naszym noclegiem i ruszyliśmy pieszo do El Buen Sabor, aby oglądając zachód słońca napić się drinka.