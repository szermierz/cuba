
\chapter[Prysznic to luksus, na który nie każdy może sobie pozwolić]{Dzień 1 - Prysznic to luksus, na który nie każdy może sobie pozwolić}

Wczoraj późnym wieczorem wylądowaliśmy w Hawanie, odebraliśmy samochód z lotniska i przyjechaliśmy do zarezerwowanego noclegu.
Wykończeni padliśmy do łóżek i obudziliśmy się dosyć wcześnie rano - a przynajmniej wcześnie według czasu lokalnego, bo w tym momencie w Polsce była godzina 11.
Dzisiejszego dnia mieliśmy w planie wybrać się na darmowe piesze zwiedzanie Hawany wraz z przewodnikiem.
Ta lokalna inicjatywa o charakterze wolontaryjnym nazywa się Free Walking Tour.
Oferują oni dwie wycieczki: po starej oraz nowej Hawanie i obie z nich chcieliśmy zaliczyć dzisiaj.
Właściciel noclegu zaoferował nam śniadanie, ale uprzejmie odmówiliśmy planując zjeść coś w centrum miasta, tuż przed wycieczką.
Umyliśmy się i wyruszyliśmy pieszo w okolice punktu zbiórki oprowadzania.
Mieszkanie, w którym przyszło nam nocować miało dosyć wysoko uniesiony sufit, zdecydowanie powyżej 3 metrów i niemalże do samego sufitu sięgające drzwi pomiędzy pomieszczeniami.
Podejrzewam, że jest to pozostałość architektoniczna utrzymywana do dzisiaj po czasach, gdy wysokie drzwi były niezbędne, aby dało się wjechać do budynku na koniu.
Pierwszą rzeczą, jaka uderzyła mnie po wyjściu z pokoju było gorąco.
Nie ma się co temu dziwić, w końcu to kraj o tropikalnym klimacie.
Niestety dalej było już gorzej.
Hawana nie zrobiła na mnie dobrego pierwszego wrażenia i moje zderzenie z odmiennością, jaką Kuba stanowi wobec Europy było brutalne.
Na ulicy paskudnie śmierdziało.
Przewijała się mieszanina czterech zapachów: najbardziej odczuwalny spalin samochodowych, kolejno wielogatunkowego gówna, dalej uryny i na koniec ostatni regularnie wyczuwalny zapach - gorzki i stęchły, niestety nie potrafię znaleźć dla niego nazwy.
Budynki w Hawanie w większości są w ruinie.
Oczywiście samo centrum z jedną galerią handlową i hotelem wygląda całkiem ładnie, ale już 500 metrów dalej do samych przedmieść rozciągają się odrapane kamienice z oknami zasłoniętymi kratami lub zabitymi deskami.
Niektóre budynki były całkowicie opustoszałe, najprawdopodobniej dlatego że mieszkanie w nich grozi niebezpieczeństwem katastrofy architektonicznej.
Kontenery na śmieci ustawiane na ulicach są dla miejscowych chyba tylko wskazówką, w okolicy czego rzucać śmieci na ziemię.
Przepraszam Cię drogi czytelniku za taki opis - tak autentycznie to wyglądało.
\par Udało nam się przylecieć na Kubie akurat na rozpoczęcie roku szkolnego.
Na ulicach często widywaliśmy grupy dzieci, stojących przed szkołą na apelu.
Uczniowie na Kubie chodzą w mundurkach.
Dzieci w szkołach podstawowych mają białe koszule i brązowy dół - chłopcy spodnie a dziewczynki spódniczki.
Dziewczynki w dodatku noszą podciągnięte wysoko białe skarpety.
Uczniowie szkół średnich mają identyczne mundurki, za to w kolorze błękitnej góry i granatowego dołu.
Im bliżej miejsca zbiórki wycieczki byliśmy, tym bardziej rozglądaliśmy się za śniadaniem.
Jak się okazało, na Kubie wszystkie sklepy otwierane są dopiero w okolicy godziny 10.
Wszystkie poza punktami dystrybucji jedzenia dla mieszkańców, bo ciężko jest mi inaczej nazwać okienko w ścianie zabite kratami z wystawionym na zewnątrz „menu” (z napisem „ofertas”) posiadającym od 1 do 4 pozycji.
Zdezorientowani otworzyliśmy ściągnięte wcześniej mapy offline, gdzie wybraliśmy restaurację w okolicy, która posiadała jakiekolwiek rekomendacje użytkowników aplikacji.
Skierowaliśmy się tam i dzięki temu, że dosyć sporo czasu zmarnowaliśmy szukając restauracji na oślep, wybiła magiczna godzina dziesiąta i restauracja okazała się otwarta.
W cenie 4 dolarów kubańskich otrzymaliśmy przystawki z owoców, w tym takich których nigdy nie jadłem - podobny do melona, tylko czerwony.
Podejrzewam, że można go kupić i u nas, ale jakoś nigdy nie wykorzystałem takiej okazji.
Pieczywo w formie jeszcze ciepłych bułeczek oraz rożków z ciasta francuskiego nadzianych mała porcyjką dżemu, smoothie z kolejnego nieznanego nam owocu i jednojajkowy omlet z bekonem, oraz całkiem dobre latte, o którego istnieniu w karcie dowiedział się ode mnie kelner.
Zadowoleni ze śniadania ruszyliśmy spokojnym spacerkiem na miejsce zbiórki wycieczki.
Turyści dzieleni są na 4 grupy według kryterium kierunku wycieczki (stara lub nowa Hawana) i języka przewodnika (angielski lub hiszpański), nas w trakcie pierwszej - po starej Hawanie - oprowadzał absolwent filologii angielskiej - rodowity Hawańczyk.
Dowiedzieliśmy się naprawdę mnóstwa ciekawych rzeczy, a przynajmniej ja uznałem całą wycieczkę za wybitnie ciekawą.
\par Nie jest wiadome jakie ludy zamieszkiwały Kubę pierwotnie.
Najprawdopodobniej była to jakaś mieszanka Azteków z innym pierwotnym ludem Ameryki środkowej, ale nie ma na to zachowanych dowodów.
Pierwotne ludy Kuby zostały wyparte przez ekspandujących się Wenezuelczyków, Dominikan i inne pobliskie nacje, które żyły tu przez około 300 lat.
Jest to pierwsza z czterech epok na Kubie, z której zachowały się dowody historyczne.
\par W 16-stym wieku na Kubę przybyli Hiszpanie, transformując ja od wschodu na zachód.
Budowali forty (głównie do ochrony przed piratami), dwory, wszystko w typowym dla siebie architektonicznym stylu.
Nazwa Habana (przez B) pochodzi od księżniczki wydanej za hiszpańskiego lorda, który się tu osiedlił.
Kuba funkcjonowała jako państwo kolonialne produkujące kakao, trzcinę cukrowa, tytoń i kawę.
Kubańczycy regularnie sprzeciwiali się wyzyskowi i pomimo nieskutecznych powstań, ciągle stawali do walki o swoją wolność. 
W okolicy 1870 roku udało im się umieścić swoich przedstawicieli w hiszpańskim rządzie, a blisko 10 lat później znieśli niewolnictwo.
\par Trzecią epokę w dziejach Kuby rozpoczyna zbrojna interwencja armii Stanów Zjednoczonych i przejęcie wyspy z rąk Hiszpan.
Powstała konstytucja, na mocy której Stany Zjednoczone mogły ingerować w sprawy kubańskie, co dopiero wiele lat później zostało zniesione, jednak silny wpływ Amerykanów pozostał aż do końca tej epoki.
Kuba nawadniana była słabym systemem irygacji do czasu, aż Kubański inżynier zaprojektował grawitacyjny podziemny system wodociągów, dzięki któremu dostępna była czysta pitna woda, co powstrzymało wiele chorób (typowych dla braku czystej wody), natomiast sam inżynier do dziś jest narodowym bohaterem, a na placu w Hawanie stoi jego pomnik otoczony fontannami.
Jest to trochę ironiczne, bo dziś Hawana boryka się z problemami z dostawą wody, przez co na ulicach ciagle widać cysterny dostarczające życiodajną ciecz do odciętych części miasta.
Kuba przez lata próbowała wyzwolić się od kolonialnego wpływu, najpierw hiszpańskiego, później amerykańskiego.
\par Rosnące niezadowolenie społeczne doprowadziło do tego, że w 1959 Fidel Castro rozpoczął rewolucję uwalniającą naród od skorumpowanego rządu marionetkowego i przynoszącą upragniony komunizm.
Taki jest punkt widzenia Kubańczyków, któremu zresztą ciężko się dziwić, biorąc pod uwagę sytuacje w jakiej byli przed rewolucją.
Uciekający prezydent ukradł całe kubańskie zapasy złota, co długofalowo w połączeniu z amerykańskim embargo, transformacją ustrojową i gospodarczą doprowadziło do silnego kryzysu, za który Kubańczycy obwiniają kapitalizm.
Otwiera to czwarta epokę w dziejach Kuby, trwająca do dzisiaj.
Kubańczycy pracują na państwowych posadach za śr.
25 dolarów kubańskich miesięcznie.
Każdy taki dolar jest wart 25 kubańskich peso.
W trakcie wycieczki uliczny rysownik namalował mi spontaniczny portret, za który jak się potem dowiedziałem dostał ode mnie 1/5 miesięcznej pensji Kubańczyka.
Posiłek „w okienku” (o których pisałem wcześniej, chodzi o te okna z kratami i cennikiem) kosztuje 3-6 peso, wygląda tak, że strach spróbować i dopiero przeraża perspektywą żywienia się nim codziennie.
Oglądaliśmy, jak Kubańczycy jedzą cierpiące od wysokiej temperatury ciastka z niezidentyfikowanym kremem, hotdogi z żółtego chleba oraz jakieś zaschnięte knuty, które podejrzewam mają skład i pochodzenie zbliżone do psich klocków.
Z okresu okupacji amerykańskiej zachowało się dużo samochodów, amerykańskich klasyków oraz zabytków infrastruktury miasta.
Oglądaliśmy przepiękną aptekę z manufakturą leków na zapleczu, odbudowaną po pożarze, niegdyś bogato wyścielona meblami i zdobieniami z mahoniu, dziś jest to jedynie pomalowana na czarno rekonstrukcja.
Następnym punktem wycieczki był postój przy rządowym banku i opis obecnej sytuacji ekonomicznej Kuby.
Kryzys postrewolucyjny doprowadził do silnej inflacji i aby przetrwać na rynku międzynarodowym wprowadzono wspomnianą wcześniej walutę kubańskiego dolara - cuc (zwanego też convertiblem) - równolegle do poprzedniej cup (kubańskiego peso).
Nową walutę kupują tylko turyści (w znaczeniu takim, że nie uczestniczy w handlu międzynarodowym. Kubańczycy również z niej korzystają), aby płacić za miejscowe towary oraz importowane produkty (jak np.
elektronikę). Dzięki temu kubański rząd dysponuje środkami, aby handlować na arenie międzynarodowej.
\par Kubańczycy korzystają z Internetu pakietowego.
Duża część Kuby pokryta jest zasięgiem Internetu radiowego, do którego można się podłączyć za pomocą pakietów Internetu kupowanych w placówkach państwowej telekomunikacji ETECSA.
Istnieje oczywiście tylko jeden prowider Internetu, więc jakiekolwiek lokalne Wi-Fi również korzysta z tego systemu i bez pakietu bajtów nie można podłączyć się do sieci, w tym także do Wi-Fi w hotelu czy restauracji.
Po wycieczce wymieniliśmy się mailami z przewodnikiem - był naprawdę sympatyczny.
Poszliśmy zjeść obiad do miejscowej restauracji, w której w menu figurują cztery potrawy: pork, chicken, fish, i customowa mieszanka trzech powyższych według życzenia klienta.
Do tego zamówiliśmy ciemne piwo (bo tylko takie było), rzekomo ichniejszej produkcji.
Byłem do tego sceptycznie nastawiony, bo nie przepadam za takim.
Okazało się natomiast wybitnie dobre i orzeźwiające!
Smakiem umieściłbym je w połowie drogi z jasnego piwa do kwasu chlebowego.
Zamówiliśmy też jedną porcję pork i jedną chicken w zamyśle takim, abyśmy oboje zjedli po połowie każdej.
Porcje nie były duże, ale smaczne i ładnie podane. Na stojaku nad talerzem wisiały nabite na pręty kawałki mięska, górujące nad ryżem i „sałatka” z marynowanych pieczarek i jakiś jakby fasolek szparagowych.
Po obiedzie wyruszyliśmy do banku wymienić resztę pieniędzy. Mi został banknot 100 euro, którego w kantorze na lotnisku nie chcieli, bo rzekomo był zbyt nowy. Kozi natomiast wymienił tam tylko część pieniędzy, bo obawiał się niekorzystnego kursu - niepotrzebnie jak się okazało, bo wszystkie kantory i banki na Kubie są państwowe i mają te same kursy.
Po drodze rozglądaliśmy się za pamiątkami w przyulicznych sklepikach, ale nieszczególnie było tam coś wartego uwagi.
Kozi poszukiwał pocztówek, w końcu gdziekolwiek na świecie tam, gdzie są turyści, są i kartki pocztowe w sklepach.
Ale nie tu.
Chyba musielibyśmy poszukać jakiejś poczty, na mapie widzieliśmy, że są tu jakieś.
Ciekawe jaka część Kubańczyków jest niepiśmienna...
kiedyś to sprawdzę!
Trafiliśmy do banku wymienić waluty, lecz jak przystało na państwowy urząd nie wymienili mi mojego banknotu, a jedynie podali adres innego banku, który może mi pomóc.
„Może” bo skąd mogę mieć pewność, że nie zostanę odurzędowany do kolejnego oddalonego o setki metrów miejsca?
Wyruszyliśmy dalej pieszo - do kolejnego banku.
Szukając kartek pocztowych i pamiątek zapędziliśmy się za daleko i przyczepił się do nas jakiś facet, który usilnie chciał nam pomoc znaleźć to czego szukamy i prowadził nas przez Hawanę - od sklepu do sklepu (z identycznym nie odpowiadającym nam asortymentem) i pytał tam czy mają to czego chcemy.
Trochę nam to było nie na rękę, bo ani nie chcieliśmy go napiwkować za pomoc, ani nie szukaliśmy nic na siłę, raczej przy okazji.
Zaprowadził nas na jakiś targ, który na całe szczęście był zamknięty, co wykorzystaliśmy, żeby się z nim pożegnać.
Zaczęły nam się udzielać obtarcia - całodniowa piesza podróż przez Hawanę niezbyt współgrała z ubranymi przeze mnie japonkami.
Dotarliśmy w końcu do drugiego banku, który jakby nigdy nic wymienił mi pieniądze.
W klimatyzowanym budynku banku odpoczęliśmy 15 minut (padł pomysł, żeby uciekać, póki nie zauważyli, że mój banknot był jakiś wadliwy, ale uznaliśmy, że jeśli urzędnik już go przyjął, to na państwowej posadzie będzie miał gdzieś to, czy wszystko było w porządku.
A przynajmniej nasz polski urzędnik tak by postąpił).
Zbliżał się termin drugiej wycieczki, więc wyruszyliśmy na miejsce zbiórki, a że udało nam się dotrzeć jakieś 20 minut przed czasem, to w barze przy samym punkcie zbiórki zamówiliśmy sobie orzeźwiające napoje alkoholowe.
Wykorzystując ostatnie minuty pozostałe przed wycieczką poszedłem skorzystać z toalety.
Drzwi do niej otwierały się ciężko - na Kubie mało co jest nowe i sprawnie działa.
A przynajmniej tak mi się wydawało (ze chodzą ciężko, druga część nadal jest prawdziwa), bo jak się okazało wewnątrz był mężczyzna siłujący się ze mną drzwiami, abym mu nie przeszkadzał w korzystaniu z toalety.
Późniejsza analiza pokazała jednak, że drzwi mają sprawną zasuwkę.
Nie można go za to winić: na Kubie mało co jest nowe i sprawne, widocznie obawiał się, zę ochroni go to mniej, niż trzymanie drzwi ręką.
\par Druga wycieczka nie była taka bogata historycznie, obejmowała nową Hawanę.
Zwiedziliśmy china town, w którym ku naszemu zdziwieniu nie było zbyt wielu mieszkańców narodowości azjatyckiej.
Taki stan rzeczy spowodowany był tym, że wynieśli się oni w trakcie rewolucji.
Kubańczycy pracują za bardzo podobne, głodowe pensje, więc prawie każdy dorabia sobie nielegalnie na boku.
Na Kubie nie ma ani jednego więźnia politycznego - każdego obywatela można przecież skazać za nielegalne praktyki.
Ulice pełne są drobnych sklepików, gdzie każdy ma jedynie kilka produktów asortymentu, np.
6 bananów i 8 mango.
Oglądaliśmy również panoramę Hawany z dachu 5-cio piętrowego budynku (popijając drink z rumu, ananasów i dżemu morelowego).
Hawana wygląda gorzej niż Częstochowa - budynki rozpadają się, ulice wyglądają źle.
Z góry natomiast wygląda to jeszcze gorzej - jak ruiny.
Dachy budynków niejednokrotnie składają się jedynie z pozostałości po najwyższej kondygnacji.
Wycieczkę skończyliśmy przy obetonowanej promenadzie przy morzu.
Wykończeni usiedliśmy na betonowej barierce i odpoczęliśmy chwilę.
Ustaliliśmy plan dalszego działania i łatwo doszliśmy do porozumienia, aby obciąć dodatkowe kilometry marszu za pomocą taksówki - nasze nogi i tak już protestowały, a jutro w planie jest wycieczka rowerowa.
Poszliśmy zatem pieszo na starą Hawanę. Tam pojechaliśmy windą jedynej miejscowej galerii handlowej na 6-ste piętro, gdzie w barze z punktem widokowym na panoramę miasta zamówiliśmy sobie kubańskie jasne piwo.
Dwa różne - spróbowaliśmy obydwu.
Kubańskie piwo jest lekkie, niskoalkoholowe, orzeźwiające.
Lekko kwaśne.
Nad Hawaną powiewał czarny dym wydzielany w ogromnej ilości przez komin jakiejś fabryki, czy raczej manufaktury.
Wśród gości tego baru był blondyn wyglądający jak Fernando Torres wraz z czarnoskórą partnerką, która wyglądała jak modelka.
Może to byli oni!
Gdy tylko usiądę przy Internecie, sprawdzę jak teraz wyglądają!
Opuściliśmy bar i wybraliśmy się do kolejnego - na drinka, bo piwo już piliśmy.
Knajpka, do której poszliśmy była kiedyś co dzień odwiedzana przez Ernesta Hemingwaya.
Przychodził do niej, zawsze takim samym krokiem kierował się w lewo, opierał o bar i prosił „to co zwykle”.
Dziś ten drink można zamówić pod nazwą „Papa Hemingway”.
Nas niestety nie wpuścili do baru, może był pełen, a może my nie byliśmy odpowiednio ubrani.
Poszliśmy zatem na parking taksówek, żeby dojechać prosto do pokoju.
Poprosiliśmy oczywiście o jazdę klasycznym Chevroletem.
Udało nam się dogadać z kierowca błękitnego klasyka.
Powiedział nam, że to model z 56-go roku, z oryginalnym silnikiem.
\par Wróciliśmy zmęczeni, obtarci i przepoceni do pokoju.
No...
i nie było wody.
Wezwaliśmy opiekuna pokoju, któremu nadaliśmy imię Gonzales (nie pamiętaliśmy, jak się nazywał, a jakoś potrzebowaliśmy o nim mówić między sobą), na pomoc.
Gonzales niestety był bezsilny, zaoferował nam za to wodę butelkowaną, za darmo.
Mieliśmy jeszcze kilka półlitrowych buteleczek zapasu, więc podziękowaliśmy, najwyżej wezwiemy go później jeszcze raz.
Umyłem obtarte stopy, twarz i zęby, co pochłonęło większość naszych zapasów wody.
Jak można się powoli domyślać - toaleta również korzystała z wody.
Jedno spłukanie zadziałało ze zgromadzonej wody, drugie wymagało użycia zewnętrznej.
Wlałem naszą ostatnią butelkę od góry, co widocznie było błędem - Kubańskie spłuczki działają chyba inaczej niż polskie.
Wlana woda nie została ani w zbiorniku, ani nie spłynęła do muszli.
Wezwaliśmy Gonzalesa z prośbą o dodatkową wodę, bo „chcieliśmy wziąć jakąś namiastkę prysznica”, tylko żeby wylać tę wodę do toalety i rozrzedzić zawarty tam materiał.
Gonzales przyniósł nam całą wodę jaką miał, prosto z lodówki.
Poradziliśmy sobie z toaletowym kryzysem, za to o ile ja mogłem się umyć letnią wodą, tak Koziemu przyszło myć się lodowatą - świeżo przyniesioną przez Gonzalesa.
Bo tą letnią wylaliśmy.
Po wszystkim przepocony położyłem się spać w nadziei, że rano będzie już woda. 

\szQuote{Prysznic to luksus, na który nie każdy może sobie pozwolić.}