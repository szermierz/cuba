
\chapter[Próba nożnego odpalenia konia]{Dzień 12 - Próba nożnego odpalenia konia}

Obudziliśmy się niezbyt wyspani, najprawdopodobniej z powodu głośno pracującej przez całą noc klimatyzacji.
Zdziwiłem się, bo specjalnie ustawiłem ją w tryb „super quiet” i na początku z tego co pamiętam działało.
Śniadanie u Esmeraldy było dosyć ubogie, ale zupełnie nam wystarczyło.
Podany sok z mango pochodził rzekomo z jej własnych zbiorów.
I faktycznie w smaku był dużo delikatniejszy, a przede wszystkim nie taki słodki jak te dotychczasowo przez nas pite.
Spakowaliśmy się, zapłaciliśmy i pożegnaliśmy.
Gdy wyjeżdżaliśmy z Matanzas udało nam się na pożegnanie stoczyć kolejny bój z gps’em.
Podejrzewam, że maszyna nie poradziła sobie z nietuzinkowym układem dróg, gdzie przez chwilę mimo prawostronnego ruchu powinniśmy wjechać na jezdnię z lewej strony, bo ruch z naprzeciwka nadjeżdżał z prawej.
Samo skrzyżowanie było do tego słabo oznaczone, przez większość czasu myśleliśmy, że wszystkie pasy w kierunku naszego wyjazdu były pod prąd.
Próbowaliśmy odnaleźć inną drogę, ale nawigacja uparcie kierowała nas na to skrzyżowanie, aż w końcu poddaliśmy się i gdy po raz kolejny do niego podjechaliśmy, udało nam się zauważyć jak inny samochód wjeżdża w nasz ukryty w tym labiryncie docelowy pas.
\par Za miastem drogi były naprawdę przyzwoite.
Jednolita, czteropasmowa nawierzchnia, bez ani jednej dziury, była czymś czego na Kubie jeszcze nie widziałem.
Myślę, że nareszcie mogę powiedzieć, że w kwestii dróg na Kubie widziałem już praktycznie wszystko.
Niestety jadąc do Hawany udało nam się też zauważyć dosyć przykre wydarzenie - zablokowaną drogę przez ustawiony poprzecznie zaprzęg.
Z daleka widzieliśmy siłującego się z wozem człowieka, więc zatrzymaliśmy się na poboczu by mu pomóc.
Gdy podeszliśmy bliżej naszym oczom okazała się przyczyna: leżący na asfalcie, ciężko oddychający koń.
Spytałem woźnicę jak możemy mu pomóc, ale odpowiedział tylko długim hiszpańskim zdaniem, z którego nie zrozumiałem ani słowa.
Na szczęście zaraz pojawili się dookoła inni i zaczęli zdejmować uprząż z konia, więc i my się przyłączyliśmy.
Po chwili siłowania z węzłami i wozem udało nam się wyswobodzić zwierzę, i wtedy jego właściciel zaczął kopać go po karku.

\szQuote{Próba nożnego odpalenia konia.}
\noindent Wyglądało to makabrycznie.
Nie wiem czy to normalne traktowanie koni na Kubie, czy ten człowiek był przerażony swoją sytuacją i nie potrafił racjonalnie do tego podejść, przez co za wszelką cenę chciał jak najszybciej opuścić jezdnię.
Koń po chwili wstał i razem z zaprzęgiem znaleźli się na poboczu, więc my wróciliśmy do auta i odjechaliśmy.
Właściciel nawet nam nie podziękował, ale zrobiła to reszta grupy pospolitego ruszenia drogowej pomocy.
\par W Hawanie zatrzymaliśmy się w północnej dzielnicy na płatnym parkingu.
Stróż powiedział nam, że cena za godzinę wynosi 1 dolar, ale za 4 dolary możemy stać tu cały dzień, bo on nigdzie się nie rusza.
Zostawiliśmy samochód pod jego opieką i poszliśmy do sklepu z rumem i cygarami - 
tego samego, który był ostatnim punktem odwiedzonym w Hawanie, gdy byliśmy tu za pierwszym razem.
Każdy z nas uzupełnił zapasy kubańskich dóbr do ilości, którą chcieliśmy zabrać do Polski.
Przed wylotem chcieliśmy jeszcze zwiedzić słynny cmentarz w Hawanie: Necrópolis de Cristóbal Colon.
Znajdował się on ponad godzinę pieszej drogi od nas, ale mieliśmy dużo czasu, więc zdecydowaliśmy się na taki spacer.
Gdy tylko weszliśmy na cmentarz, pierwszą rzeczą do jakiej zostaliśmy zawołani, była kasa biletowa.
Chyba za bardzo wyglądaliśmy jak turyści, bo nie sądziłem, że odwiedziny cmentarza są płatne dla każdego.
Jak się okazało, opłata była nie tyle biletem wstępu, co wykupieniem sobie oprowadzania z przewodnikiem i na szczęście w kanciapie przewodników znalazł się nawet jeszcze jeden mówiący po angielsku.
\par Cmentarz Colon jest największym w Hawanie.
Rozrósł się najbardziej za czasów plag i chorób dziesiątkujących miasto, dlatego też przynajmniej wtedy znajdował się na jego przedmieściach, a nie w centrum.
Początkowo grobowce tutaj posiadali jedynie arystokraci i baronowie, za to dziś, aby zostać tu pochowanym - jak to określił przewodnik - wystarczy być martwym.
Bogate hiszpańskie rody rywalizowały między sobą stawiając coraz to wyższe pomniki, co było świadectwem ich bogactwa.
Same groby były bardzo różne, na cmentarzu są między innymi dwie piramidy o wysokości około 3 metrów.
W jednej z nich pochowany jest profesor architektury Uniwersytetu Hawańskiego.
W czasach kolonialnych Kubańczycy handlujący cukrem byli bogaci, ale nie mieli żadnego tytułu, więc nie mogli zostać pochowani wśród arystokratów.
Niektórzy kupowali tytuły od ubogich Hiszpanów, co dawało im równą pozycję i miejsce na cmentarzu.
Między innymi w podobny sposób pochowany został barman Ernesta Hemingwaya, który swoją fortunę uzbierał z napiwków zbieranych od gości z całego świata chcących napić się u tej samej osoby, co słynny pisarz.
Najwyższy pomnik postawiony został ofiarom wypadku z 17 maja 1890 roku.
Wybuchł pożar w magazynie materiałów wybuchowych i do działania wysłana została lokalna straż pożarna.
Strażakom powiedziano, że akcja jest bezpieczna, natomiast po pewnym czasie nastąpiła ogromna eksplozja, która pociągnęła za sobą od 21 do 30 ludzi w młodym wieku - dokładna liczba nie jest znana.
Pomnik jest wykonany z najdroższego marmuru na świecie i posiada bogate i symboliczne zdobienia, takie jak nietoperze oznaczające ślepotę w dymie palącego się prochu czy łzy wiszące na poręczach.
\par Po skończeniu zwiedzania poszliśmy do kościoła w centrum cmentarza.
Usiedliśmy na ławce i podziwialiśmy piękne witraże, których szkło było celowo przyciemnione tak, by zbudować odpowiedni kontrast ekspozycji.
Dzięki temu powstało wrażenie jakby wyszczególnione elementy witraży aż błyszczały.
Nacieszyliśmy się tym widokiem i wyszliśmy z cmentarza.
Niechętnie zaopatrując się na kolejny godzinny spacer w wysokiej temperaturze, postanowiliśmy wrócić taksówką.
Aby nie nadziać się na wysokie ceny, zdecydowaliśmy się pojechać z kierowcą zaparkowanej przy cmentarzu Łady (z logiem taxi za szybą).
Pokazaliśmy mu na mapie, gdzie chcemy się dostać i po chwili już jechaliśmy.
Kierowca spytał nas skąd jesteśmy, a gdy usłyszał odpowiedź, zaczął dopytywać jak nam się żyje w Polsce bez socjalizmu.
Nie trwało to długo, aż zaczął marudzić na kubański ustrój, nawet wspomniał, że nie może się doczekać aż w końcu będą mieli inny.
Spytał na koniec jakie mamy w Polsce samochody, a ja stoczyłem wewnętrzną walkę, żeby to w sobie stłumić, ale oczywiście przegrałem i pokazałem mu zdjęcie mojego auta.
Zamknęło to temat rozmów o samochodach.
Na pożegnanie zapłaciliśmy taksówkarzowi, a ja dałem mu na pamiątkę złotówkę.
Oby służyła mu jako talizman na szczęście.
\par W centrum zjedliśmy obiad, a właściwie Kozi zjadł, bo ja przed podróżą nic nie jadam.
Wypiłem tylko lemoniadę z lodem, za to Kozi zamówił sobie makaron z sosem bolognese.
Po posiłku wróciliśmy pieszo do samochodu i pojechaliśmy na lotnisko.
Gdy dojechaliśmy na parking samochodów wynajmowanych, spojrzeliśmy na licznik. Wypożyczając auto spisaliśmy w umowie jego stan, przez co wiedzieliśmy dokładnie, ile kilometrów pokonaliśmy. Nasz trzeci kompan pozwolił nam przebyć łącznie trasę 3003 km.
Stojąc na parkingu spakowaliśmy kupione tego samego dnia butelki rumu tak, aby możliwie zabezpieczyć je w transporcie, po czym zabraliśmy swoje bagaże i ruszyliśmy oddać samochód, którym Kubę objechaliśmy dookoła.
W Internecie krążą plotki, że Kubańczycy starają się na siłę znaleźć jakieś wady w samochodzie, żeby uszczknąć na kaucji.
Nas to akurat nie spotkało - może dlatego, że samochód był prawie nowy, przez co dostaliśmy go bez wad, no i oddaliśmy również bez, był jedynie brudny z wszechobecnej na Kubie pomarańczowej ziemi.
Na lotnisku wymieniłem otrzymaną kaucję na euro i razem z Kozim usiedliśmy, czekając na otworzenie bramek do odprawy.