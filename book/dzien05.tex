
\chapter[Tak dobrze żarło, a zdechło]{Dzień 5 - Tak dobrze żarło, a zdechło}

Dzisiejszy dzień przywitał mnie zaskoczeniem: komary w Cienfuegos gryzą.
I to do tego całkiem obficie!
Spod kołdry wystawały mi jedynie stopy, ale to wystarczyło komarom pokąsić mnie w blisko 7 miejscach.
Z samego rana Kozi zadzwonił na infolinie banku i odblokował swoją kartę, więc zaplanowaliśmy wypłacić pieniądze w Trinidadzie.
Śniadanie u Rejczel było najmniej obfite w owoce ze wszystkich do tej pory, za to wyróżniało się pysznym tostem z szynką i serem z opiekacza.
Podano nam również cały dzban smoothie z guawy, więc napoju tego mieliśmy pod dostatkiem.
\par Przed wyjazdem chciałem jeszcze powyrzucać plastikowe butelki, które zostały nam po wodzie.
Pokój u Rejczel był pierwszym, który posiadał mały aneks kuchenny i gdy szukałem śmietnika, potwierdził moje obawy: posiadanie śmietnika pod zlewem jest typowo polskim zwyczajem, nie kubańskim.
Nasze plastikowe śmieci powędrowały do jedynego kubełka: w łazience.
Zapłaciliśmy, spakowaliśmy torby do auta i pojechaliśmy do Trinidadu.
Żaden ze mnie botanik, ale zauważyłem, że zdecydowana większość kubańskiej roślinności liściastej stanowią gatunki z malutkimi listkami.
Palmy z długimi wąskimi liśćmi, drzewa z listkami przypominającymi te z jarzębiny.
Podejrzewam, że taki kształt liści pozwala roślinie zatrzymać dużo więcej wody, a raczej od drugiej strony: obszerne liście o dużej powierzchni powodują jej dużą utratę.
Nie ma tu za to typowo polskiej roślinności iglastej, czego idąc tym tropem nie potrafię już wytłumaczyć.
Na pewno istnieje ku temu jakiś powód i być może jest nim moje błędne rozumienie botaniki.
Koniecznie sprawdzę to po powrocie do domu!
Zapomniałbym wspomnieć: kilkukrotnie widziałem też względnie niskie palmy z ogromnymi liśćmi - wyjątek od tej reguły.
\par W drodze do Trinidadu Kozi zwrócił mi uwagę, że na gps’ie mijamy właśnie jakieś jezioro i jeśli skieruje obiektyw w prawo będę mógł uchwycić je na zdjęciu.
To całkiem zabawne, bo tym jeziorem było Morze Karaibskie.
I rzeczywiście wyglądało prześlicznie.
W trakcie dzisiejszej jazdy zostaliśmy po raz pierwszy na Kubie zatrzymani przez policję.
Oficer stojący przy drodze machnął na nas ręka, a my zjechaliśmy na pobocze i czekaliśmy aż podejdzie, zgodnie z procedurą naszego kraju (nie znamy żadnej innej!).
Po chwili zwątpiliśmy, wrzuciliśmy wsteczny i podjechaliśmy do dwóch policjantów.
Skręciłem szybę i próbowałem się dowiedzieć jaki był powód zatrzymania.
Policjant wydobywał z siebie dźwięki jakby starał się odpowiedzieć mi po angielsku, ale akurat zapomniał wszystkie słowa.
Po chwili zmagań samego ze sobą odpowiedział: „no problem, continuar”.
O ile to, że miejscowi machają nam, widząc samochód z turystami, tak machający policjant był nieco wprowadzający w błąd.
Może chodziło o to, że nie mieliśmy zapalonych świateł?
Chyba już nigdy się nie dowiemy.
\par Trinidad nie okazał się najpiękniejszym miastem na Kubie.
Nie był też najbrzydszym (Hawana jest w rzeczy samej niezrównana).
Zaraz po przyjechaniu zatrzymaliśmy się na stacji benzynowej, ponieważ jest piątek i niewykluczone, że kolejną okazję zatankować możemy mieć dopiero w poniedziałek.
Kolejka do budki opłat składała się z trzech osób włącznie z Kozim, a jednak czas oczekiwania wyniósł nas blisko 20 minut.
W takich chwilach zawsze zastanawiam się, jak to jest możliwe, że załatwiając sprawę w polskim urzędzie każda osoba stoi przy okienku 20 minut, a gdy nadchodzi moja kolej, to obsługa zajmuje 30 sekund.
Co Ci ludzie robią przy tych okienkach?
To chyba jakaś komunistyczna umiejętność, której moje pokolenie już nie pozna.
Kubańczycy natomiast opanowali ją bardzo dobrze.
W poszukiwaniu noclegu pojechaliśmy na przedmieścia leżące przy morzu.
O ile miasto wyglądało jak plątanina niebezpiecznych dzielnic, tak przedmieścia były jeszcze gorsze.
Zrezygnowaliśmy i wróciliśmy do centrum Trinidadu kierując się mapami offline, w których wypatrywaliśmy punktów z wysokimi ocenami. Nie zawsze nam się to sprawdzało, ale przynajmniej był to jakikolwiek punkt wyjścia w poszukiwaniach.
Wybraliśmy możliwie wysoko oceniony nocleg i podjechaliśmy tam, zaparkowaliśmy auto na ulicy i dalej ruszyliśmy pieszo.
Jak się okazało - w tym miejscu nie było w ogóle noclegu, a jedynie restauracja.
W dodatku pod zupełnie inną nazwą.
Ponownie otworzyliśmy mapę i skierowaliśmy się do kolejnej dobrze ocenionej lokalizacji w okolicy.
Tym razem okazało się, że pokój faktycznie istnieje i w dodatku jest wolny - jedynie został dopiero co zwolniony więc właśnie jest sprzątany.
Właścicielka poczęstowała nas herbatą, aby umilić nam oczekiwanie.
Nadaliśmy jej imię Dolores.
\par Nasz pokój znajdował się na piętrze i w środku był śliczny!
Na półce stała mała skrzyneczka pełna muszelek.
Udało mi się wygrzebać z portfela 1 polski grosz, bity w 2019 roku, który schowałem w skrzyneczce.
Może kiedyś ktoś go odnajdzie i zabierze ze sobą jako talizman.
Zarówno widok z okna pokoju, jak i tarasu był świetny!
Dachy Trinidadu prezentują się zdecydowanie lepiej niż ulice.
Oboje wzięliśmy prysznic i ruszyliśmy w kierunku banku.
Jak się okazało - nadal nie mogliśmy wypłacić pieniędzy z bankomatu.
Kozi zaproponował, abyśmy weszli do oddziału banku i wypłacili przy okienku.
Sam nie wierzyłem, że to się uda, ale może pracownik banku nakieruje nas co jest nie tak.
Pojawił się tylko mały szkopuł - ochroniarz nie wpuścił nas do banku.
Być może był to bank nie oferujący swoich usług turystom.
Odszukaliśmy na mapie bank o możliwie miłobrzmiącej dla nas nazwie: Banco de Financiero International, i tam właśnie się udaliśmy.
Usiedliśmy w klimatyzowanym pomieszczeniu i oczekiwaliśmy na swoje miejsce w kolejce.
Gdy już się doczekaliśmy, wypłata oczywiście się nie powiodła, nie niosąc w dodatku żadnych wskazówek.
Ja czułem się spokojny - zawsze możemy skontaktować się z kimś z Polski, aby przelać pieniądze z konta Koziego na moje i wypłacić.
Przelew taki musiałby tylko być ekspresowym, bo oczekiwanie na środki spauzowałoby nam podróż.
Kiedy wychodziliśmy ochroniarz banku zalecił nam, abyśmy poszli do punktu telekomunikacyjnego i kupili kartę Internetową.
Karta taka pozwala podłączyć się do jednej z wielu lokalnych sieci wifi (a następnie zalogować danymi z karty).
Tak też zrobiliśmy i ku naszemu zdziwieniu karta kosztowała tylko 2 kubańskie dolary.
Łącznie dzwoniąc na infolinię koszt połączeń wyniósł już jakieś 150zł, za to kartę można było też odblokować przez Internet w cenie 8zł.
Rachunek ten odbił się złością na Kozim.
Posiadając jednak kartę Internetową Kozi wykonał z telefonu przelew środków na swoje drugie konto w obrębie tego samego banku, więc natychmiastowo.
Następnie poszliśmy znów spróbować szczęścia w Banco Internacional.
Gdy już siedzieliśmy w środku oczekując na swoją kolej, światła w banku pogasły (ale nie było ciemno, w końcu to środek dnia).
Pomyślałem sobie, że mamy strasznego pecha - jak już ma nam się udać, to akurat trafi się awaria prądu.
Kozi skomentował to tylko kwestią:

\szQuote{„Tak dobrze żarło, a zdechło”}

Minęło może 10 minut i zasilanie wróciło.
Światła znów się zapaliły, a bank powoli wracał do życia.
Nadal trochę to trwało, w końcu cała infrastruktura komunikacji finansowej musi powstawać.
Jedyną rzeczą, która nadal nam się nie udała była wypłacona kwota, bowiem tylko część zamierzonych pieniędzy bank pozwolił nam wybrać. Wynikało to pewnie z limitu na karcie, ale możemy codziennie wypłacać kolejna część, więc naszej podróży nic już nie zagraża.
Świętując zażegnanie kryzysu wybraliśmy się na obiad do wysoko ocenionej (w rankingu map offline) restauracji.
Wnętrze było ślicznie urządzone, z dużą dozą roślinności.
Obok naszego stolika stał drewniany moździerz do kawy - pilon.
Już na wejściu zamówiliśmy drink będący specjalnością domu, aby nie czekać na niego aż przewertujemy menu.
Okazał się nim niezmieszany sok pomarańczowy z rumem.
To połączenie tak przypadło mi do gustu, że napisałem je sobie na liście drinków, które chce nauczyć się robić.
Na obiad zamówiłem sobie zupę serową, szarpaną wieprzowinę w sosie pomidorowym i karmelowy deser.
Kozi natomiast poprosił o zupę pomidorową, wołowe vindaloo i waniliowe lody.
Cały posiłek reprezentował bardzo wysoki poziom i nie dziwię się, że restauracja miała taką wysoką ocenę.
Lody wprawdzie były w połowie waniliowe, a w połowie kokosowe, ale nie określiłbym tego jako wadę.
Mój karmelowy deser z kolei okazał się puddingiem zatopionym w karmelu.
\par Z napełnionymi brzuchami ruszyliśmy obejrzeć Trinidad z punktu widokowego, który znajdował się na szczycie lokalnej wieży.
U jej stóp okazało się, że jest to muzeum.
Kupiliśmy więc bilety i powoli rozpoczęliśmy zwiedzanie.
Owe muzeum posiadało wystawy zdjęć przedstawiających wydarzenia jak mniemam związane ze szkolnictwem.
Są to jednak tylko domysły, bo zdjęcia nie były podpisane, a jedynie przy niektórych widniała data.
Najstarsze zdjęcia datowane były rokiem 1905, więc początkiem ery okupacji Stanów Zjednoczonych.
Najwięcej zdjęć było z czasów przed samą rewolucją, w latach około 1950-1958, więc podejrzewam, że Trinidad mógł odgrywać istotną rolę w całym procederze.
Weszliśmy w końcu schodami na szczyt wieży.
Wieża ta wyposażona była w dzwon o około metrze wysokości oraz metrze średnicy.
Wydaje mi się, że cały budynek był Kubańską szkoła, a dzwon ten służył do wzywania dzieci do nauki.
Gdy już nacieszyliśmy się tym widokiem zaczęliśmy schodzić na dół schodami, aby dotrzeć do ostatniego punktu naszego dzisiejszego planu: lokalnego drinka.
W Trinidadzie lokalnym specjałem jest drink na bazie soku z limonki i miodu, z dodatkiem przedestylowanego tylko raz rumu.
Nazywa się canchanchara, a my poszliśmy napić się go do miejsca o tej samej nazwie - baru Canchanchara.
Napoje podali nam w glinianych miseczkach, z patyczkiem do rozśmieszania miodu.
Soku było tak dużo, że nawet dokładne rozpuszczenie miodu nie powodowało, że drink robił się słodki.
Był za to bardzo orzeźwiający i podejrzewam, że to o to właśnie chodziło.
W samym barze Canchanchara po podłożu z kamieni biegały luzem jaszczurki.
Kozi naliczył ich 5, ale kto wie, ile z nich jeszcze tam było i odpoczywało w ukryciu.
Zdecydowaliśmy się dodać jeden punkt do dzisiejszego programu: wejście na kolejny, tym razem najwyższy, punkt widokowy: Loma de la Vigia - górę piętrzącą się 180m ponad miastem.
Ruszyliśmy zatem w ponad 30sto stopniowym upale pnąc się po ścieżce prowadzącej na szczyt.
Obrany szlak był dosyć prosty, ale temperatura i spory rozmiar obiadu niesionego w żołądku doprowadził mnie do stanu, w którym aby ukończyć wspinaczkę, musiałem ratować się nakryciem głowy z ręcznika zmoczonego wodą.
\par Na szczycie znajdowało się obserwatorium.
Gdy tylko zbliżyliśmy się do jego bramy ukazał się nam przewodnik, który zaoferował chłodne napoje i zaprowadził nas na punkt widokowy na dachu.
Żaden widok na Kubie nie był tak spektakularny jak ten.
Zrobione tu zdjęcia wyszły pięknie, ale żadne z nich nie oddało nawet w 50\% tego, co widziały oczy.
Z jednej strony widać było jak na dłoni lokalne plantacje i manufaktury.
Za nimi z kolei rozciągały się łańcuchy górskie.
Z drugiej strony leżał Trinidad.
Miasto z tej perspektywy mieniło się mnóstwem kolorów.
Za nim z kolei błyszczało Morze Karaibskie.
Wykończeni wchodzeniem zdecydowaliśmy się zejść bardzo powoli.
Nigdzie się nie spieszyliśmy, była całkiem młoda godzina, a my zrealizowaliśmy już plan całego dnia.
Po powrocie do pokoju wziąłem chłodny, orzeźwiający prysznic.
W każdym z pokoi, w których byliśmy do tej pory była lodówka z piwem.
Oczywiście płatnym, ale dzięki temu, że znajdowała się w pokoju, nie trzeba było iść po nie aż do sklepu.
Pozostała jedynie kwestia tego, czy napić się w klimatyzowanym pokoju, czy na dusznym tarasie.
Cóż...
wypić piwo w chłodnym pokoju możemy w domu, a na tarasie z widokiem na dachy Trinidadu tylko tutaj.