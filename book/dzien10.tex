
\chapter[Czasem łatwiej jest dorobić drogę obok, niż załatać dziurę w moście]{Dzień 10 - Czasem łatwiej jest dorobić drogę obok, niż załatać dziurę w moście}

Moskitiera w oknach okazuje się zupełnie nieskuteczna, jeśli wcześniej wpuści się komary przez drzwi.
Na całe szczęście dla mnie nasz pokój posiadał dużo lepsze zabezpieczenie: Koziego.
Komary przez noc doszczętnie go pogryzły, mnie zostawiając w spokoju.
Gdy wychodziliśmy na śniadanie policzyłem 9 bąbli na jednej jego ręce.
Śniadanie u Johnnego było pierwszym od dawna, gdzie do stolika podano nam mleko.
Ucieszeni rzuciliśmy się na nie.
Ja zrobiłem sobie kawę z mlekiem, Kozi z kolei upragnione kakao.
Na stole brakowało jakiejkolwiek formy szynki, a do picia podany nam był sok z ananasa.
Poza tym właściwie nie różniło się od pozostałych śniadań na Kubie.
Zapłaciliśmy za nocleg i pożegnaliśmy się z Johnnym, po czym odjechaliśmy na zachód.
Kolejny nocleg nie był zaplanowany w prywatnie wynajętym pokoju, a w hotelu.
Jesteśmy w końcu na wakacjach.
\par Wyjeżdżając z La Boca, udało nam się zatankować na pierwszej napotkanej stacji benzynowej.
Zarówno my, jak i Kubańczycy jeździmy bez włączonych świateł.
W dzień nie są tu wymagane, za to w nocy są niezbędne, aby omijać dziury w drodze - co nas nie dotyczy, ponieważ w nocy nie jeździmy.
Nie jestem pewien czy na Kubie wymagane jest posiadanie czerwonych świateł stopu z tylu pojazdu.
Oczywiście każdy samochód posiada tu takie światła, ale niejednokrotnie nadzialiśmy się na to, że nie świeciły.
Wydarzyło się to między innymi w Chambas, gdzie ledwo wyhamowaliśmy przed cysterną, która nagle zatrzymała się zarówno bez kierunkowskazu, jak i świateł stopu.
Ruch drogowy nie jest tu duży, ale Kubańczycy ponownie stają na wysokości zadania i też mają swoje zagrożenia na drodze.
W Yaguajay zatrzymaliśmy się na stacji benzynowej, aby uzupełnić paliwo.
Podczas gdy Kozi tankował, ja miałem okazję podziwiać zmagania kierowcy najprawdopodobniej 70cio letniego niebieskiego Pontiaca.
Podjechał on pod dystrybutor z dieslem, skręcił szybę, a następnie sięgnął na zewnątrz ręką, aby otworzyć sobie drzwi kierowcy.
Następnie wysiadł, a puszczone drzwi bezwładnie oparły się o nadwozie samochodu, wisząc tak krzywo, że ciężko było mi uwierzyć, że przed chwila zamykały konstrukcje.
Kierowca wrócił i otworzył bagażnik.
Jak się okazało, pomimo widocznego wlewu paliwa z boku samochodu, wlewa się je do bagażnika.
Następnie wrócił do okienka zapłacić, po czym wsiadł do auta, magicznie zamknął niedopasowane drzwi odpowiednim trzaśnięciem i przystąpił do uruchamiania silnika.
Proces ten polegał na ciągłym powtarzaniu cyklu: zapłon, około sekunda pracy, zgaśnięcie.
Po przepracowaniu 12 cykli silnik uruchamiał się na stałe i maszyna pozwoliła odjechać ze stacji.
Szczerze mówiąc i tak jestem pod wrażeniem, że pojazd nadal jeździ.
Co więcej: taksówki w Hawanie dają radę bez tych wszystkich cyrków, a też mają jakieś 70 lat.
Po paru godzinach dojechaliśmy w okolice Caibarién, skąd odbiliśmy na północ, na wyspę Cayo Santa Maria.
Wyspa ta połączona jest drogą położoną na usypanym pasie lądu.
Trasa ta mierzy blisko 50km.
Wjeżdżając na nią przeszliśmy kontrolę paszportów i musieliśmy uiścić opłatę w wysokości 2 dolarów.
Cena ta wydaje się śmiesznie mała w porównaniu z naszą polską autostradą na linii Katowice-Kraków.
\par W sierpniu 2017 roku w Kubę uderzył huragan Irma, który punktowo zniszczył tę trasę.
Została ona w zdecydowanej większości naprawiona, ale niektóre jej fragmenty wykluczone były z użycia do dzisiaj.
Między innymi jeden z mostów, na którego środku była spora dziura. Pewnie moglibyśmy ją ominąć, ale raczej nie byłoby to zbyt bezpieczne, bo taki wyłom najprawdopodobniej nadwyrężył strukturę konstrukcji.
Kubańczycy rozwiązali ten problem dookoła - dosłownie.
Obok mostu usypany był łuk, na którym zbudowano drogę.

\szQuote{Czasem łatwiej jest dorobić drogę obok, niż załatać dziurę w moście.}
\noindent Naszym hotelem okazał się być 5cio gwiazdkowy Playa Cayo Santa Marta.
Zajechaliśmy na miejsce i wysiedliśmy z samochodu.
Już po chwili podbiegł do nas stróż parkingu z daleka gwiżdżąc na nas, abyśmy zwrócili na niego uwagę.
Kubańczycy mają paskudny zwyczaj gwizdać na ludzi, od których czegoś chcą.
My musieliśmy przeparkować, oczywiście na parking dla gości znajdujący się około 150 metrów dalej.
Gdy już się z tym uporaliśmy, stawiliśmy się na recepcji, gdzie opłaciliśmy pobyt.
W cenie jednego noclegu wliczony był obiad, kolacja, jutrzejsze śniadanie i kolejny obiad - godzinę po wymeldowaniu.
Hotel składał się z kompleksu trzypiętrowych bloków mieszkalnych, basenów, barów, punktów usług.
Nie chcę rozpisywać się o tym jak spędziliśmy czas w hotelu, bo nie wzbogaci to kulturowo czy merytorycznie tego dziennika.
Wspomnę tylko, że korzystaliśmy z usług spa, barów i restauracji, a mi przyszło w końcu zjeść na Kubie pizzę.
\par Wieczorem, gdy siedzieliśmy w barze cygarowym, przysiadł się do nas mężczyzna szukający pogawędki.
Przedstawił się imieniem Francis i dodał, że mieszka w Kanadzie, w Quebec.
Po chwili dołączył do nas jego kolega, z którym we dwóch tu przylecieli: Fred.
Niezwykle przyjemnie rozmawiało nam się z nimi na tematy tak ogólne jak życie, jego filozofia, czas wolny i poczucie młodości.
Oboje byli w hotelu na wakacjach ze swoimi żonami, więc niedługo musieli zbierać się do pokoi.
Francis podarował mi na pożegnanie otwieracz do butelek ze swoim imieniem - dziękuje!
Zgodnie z tym jaki stereotyp krąży po Internecie, Kanadyjczycy to naprawdę mili i uprzejmi ludzie.
A przynajmniej ta dwójka całkowicie pasuje do tego opisu.
Dopaliliśmy swoje cygara i spacerem udaliśmy się nad morze, gdzie usiedliśmy na piaszczystej plaży i pijąc zimne piwo kontynuowaliśmy rozmowy o życiu.